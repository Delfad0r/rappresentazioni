\chapter{Algebre e Moduli}

\section{Definizioni}

\begin{definition}
Sia $\mathbb{K}$ un campo. Si dice \emph{algebra su $\mathbb{K}$} (o $\mathbb{K}$-algebra) uno spazio vettoriale $A$ su $\mathbb{K}$ dotato di un prodotto bilineare $\cdot:A\times A\to A$ tale che $(A,+,\cdot)$ sia un anello. $A$ si dice \emph{finita} se ha dimensione finita come $\mathbb{K}$-spazio vettoriale.
\end{definition}

\begin{remark}
Sia $\mathbb{K}$ un campo, $A$ una $\mathbb{K}$-algebra non nulla. Consideriamo la mappa $\varphi:\mathbb{K}\to A$ definita da $\varphi(a)=a1_A$. Si verifica facilmente che $\varphi$ è un omomorfismo di anelli iniettivo. Inoltre, dati $a\in\mathbb{K}\comma x\in A$ vale
$$
\varphi(a)x=ax=x(a1_A)=x\varphi(a),
$$
dunque $\im\varphi\subseteq Z(A)$. Viceversa, dato un anello $A$ e un omomorfismo di anelli non nullo $\varphi:\mathbb{K}\to Z(A)$, $A$ acquista una struttura di $\mathbb{K}$-algebra, con il prodotto per scalare definito da $ax=\varphi(a)x$.
\end{remark}

\begin{definition}
Siano $A$ e $B$ $\mathbb{K}$-algebre. Si dice \emph{omomorfismo di algebre} un'applicazione lineare $\varphi:A\to B$ che sia anche un omomorfismo di anelli. $\varphi$ si dice \emph{isomorfismo di algebre} se è un omomorfismo di algebre biiettivo.
\end{definition}

\begin{definition}
Sia $\mathbb{K}$ un campo, $G$ un gruppo, $\mathbb{K}[G]$ lo spazio vettoriale libero su $G$ con base $\{e_g\}_{g\in G}$. Si dice \emph{algebra di gruppo} lo spazio vettoriale $\mathbb{K}[G]$ dotato del prodotto $e_ge_h=e_{gh}$ esteso per bilinearità.
\end{definition}

\begin{definition}
Sia $R$ un anello. Si dice \emph{$R$-modulo sinistro} un gruppo abeliano $(M,+)$ dotato di un prodotto $\cdot:R\times M\to M$ che soddisfa le seguenti proprietà.
\begin{enumerate}[(i)]
\item $(a+b)x=ax+bx$ per ogni $a,b\in R\comma x\in M$.
\item $a(x+y)=ax+ay$ per ogni $a\in R\comma x,y\in M$.
\item $1_Rx=x$ per ogni $x\in M$.
\item $(ab)x=a(bx)$ per ogni $a,b\in R\comma x\in M$.
\end{enumerate}
\end{definition}

Gli \emph{$R$-moduli destri} si definiscono analogamente, ma con il prodotto a destra (ovvero $\cdot:M\times R\to M$).

Dato un $R$-modulo $M$, potremo scrivere $_RM$ in luogo di $M$ per evidenziare il fatto che $M$ è un $R$-modulo sinistro, e $M_R$ per evidenziare il fatto che si tratta invece di un $R$-modulo destro.

\begin{definition}
Sia $(R,+,\cdot)$ un anello. Si dice \emph{anello opposto} l'anello $(\op{R},+,\op{\cdot})$, dove $a\op{\cdot}b=b\cdot a$.
\end{definition}

È evidente che gli $R$-moduli sinistri sono in corrispondenza biunivoca con gli $\op{R}$-moduli destri. Nel seguito ci limiteremo dunque a enunciare i risultati per moduli sinistri, intendendo implicitamente che valgono enunciati analoghi per i moduli destri.

\begin{definition}
Siano $M,N$ $R$-moduli sinistri. Un'applicazione $f:M\to N$ si dice \emph{omomorfismo di moduli} se è $R$-lineare, ovvero se
$$
f(ax+by)=af(x)+bf(y)
$$
per ogni $a,b\in R\comma x,y\in M$. $f$ si dice \emph{isomorfismo di moduli} se è $R$-lineare e biiettiva.
\end{definition}


\begin{definition}
Sia $A$ una $\mathbb{K}$-algebra, $M$ un $A$-modulo sinistro. Si dice \emph{algebra degli endomorfismi di $M$} 
\end{definition}



\section{Quozienti, Prodotti e Somme}

\begin{definition}
Sia $M$ un $R$-modulo sinistro. Sia $N\le M$ un sottogruppo. Diciamo che $N$ è un \emph{sottomodulo} di $M$ (e scriviamo $N\le M$) se per ogni $a\in R$ vale $aN\subseteq N$.
\end{definition}

\begin{proposition}\thlabel{module-homomorfism-ker-im-submodule}
Siano $M,N$ $R$-moduli sinistri, $f:M\to N$ un omomorfismo di moduli. Allora $\ker f\le M$ e $\im f\le N$.
\end{proposition}
\begin{proof}
Si tratta di semplici verifiche.
\end{proof}

\begin{definition}
Sia $M$ un $R$-modulo sinistro, $N\le M$ un sottomodulo. Si dice \emph{quoziente di $M$ per $N$} un $R$-modulo sinistro (indicato con $M/N$) dotato di un omomorfismo di moduli $\pi:M\to M/N$ con $N\le\ker\pi$ che soddisfa la seguente proprietà universale: per ogni $R$-modulo sinistro $L$ e per ogni omomorfismo di moduli $f:M\to L$ con $N\le\ker f$ esiste un unico omomorfismo di moduli $\bar{f}:M/N\to L$ che fa commutare il diagramma
$$
\begin{diagram}
M\arrow{r}{f}\arrow[swap]{d}{\pi}&L\\
M/N\arrow[swap]{ru}{\bar{f}}
\end{diagram}
$$
ovvero tale che $f=\bar{f}\circ\pi$.
\end{definition}

Si dimostra facilmente che il modulo quoziente è unico a meno di isomorfismo canonico, e si può costruire quozientando $M$ per la relazione di equivalenza $x\sim y\iff x-y\in N$.

\begin{definition}
Siano $\{M_i\}_{i\in I}$ degli $R$-moduli sinistri. Si dice \emph{prodotto diretto degli $M_i$} un $R$-modulo sinistro $M$ dotato di omomorfismi di moduli (proiezioni) $\pi_i:M\to M_i$ che soddisfa la seguente proprietà universale: per ogni $R$-modulo sinistro $N$ e per ogni famiglia di omomorfismi di moduli $\{\varphi_i:N\to M_i\}_{i\in I}$ esiste un unico omomorfismo di moduli $\varphi:N\to M$ che per ogni $i\in I$ fa commutare il diagramma
$$
\begin{diagram}
N\arrow{r}{\varphi_i}\arrow[swap]{rd}{\varphi}&M_i\\
&M\arrow{u}{\pi_i}
\end{diagram}
$$
ovvero tale che $\varphi_i=\pi_i\circ\varphi$.
\end{definition}

Il prodotto diretto degli $M_i$ si indica con $\prod_{i\in I}M_i$. Si dimostra facilmente che il prodotto diretto è unico a meno di isomorfismo canonico, e si può costruire prendendo il prodotto cartesiano degli $M_i$.


\begin{definition}
Siano $\{M_i\}_{i\in I}$ degli $R$-moduli sinistri. Si dice \emph{somma diretta degli $M_i$} un $R$-modulo sinistro $M$ dotato di omomorfismi di moduli (immersioni) $\iota_i:M_i\to M$ che soddisfa la seguente proprietà universale: per ogni $R$-modulo sinistro $N$ e per ogni famiglia di omomorfismi di moduli $\{\varphi_i:M_i\to N\}_{i\in I}$ esiste un unico omomorfismo di moduli $\varphi:M\to N$ che per ogni $i\in I$ fa commutare il diagramma
$$
\begin{diagram}
M_i\arrow{r}{\varphi_i}\arrow[swap]{d}{\iota_i}&N\\
M\arrow[swap]{ru}{\varphi}
\end{diagram}
$$
ovvero tale che $\varphi_i=\varphi\circ\iota_i$ per ogni $i\in I$.
\end{definition}

La somma diretta degli $M_i$ si indica con $\Dirsum_{i\in I}M_i$. Si può dimostrare che le immersioni $\iota_i$ sono iniettive, dunque possiamo sempre immaginare che gli $M_i$ siano sottomoduli di $\Dirsum_{i\in I}M_i$. Se $I$ è finito (che è l'unico caso di cui ci occuperemo) vale
$$
\Dirsum_{i\in I}M_i\iso \prod_{i\in I}M_i
$$


\begin{definition}
Sia $M$ un $R$-modulo sinistro, $\{M_i\}_{i\in I}$ sottomoduli di $M$. Si dice \emph{somma degli $M_i$} (e si indica con $\sum_{i\in I}M_i$) il più piccolo sottomodulo di $M$ che contiene tutti gli $M_i$.
\end{definition}

\begin{definition}
Sia $M$ un $R$-modulo sinistro, $\{M_i\}_{i\in I}$ sottomoduli di $M$. Consideriamo l'unico omomorfismo di moduli $\Phi:\Dirsum_{i\in I}M_i\to\sum_{i\in I}M_i$ tale che $\Phi(x)=x$ per ogni $i\in I\comma x\in M_i$. Evidentemente $\Phi$ è suriettivo; se è anche iniettivo si dice che gli $M_i$ sono \emph{in somma diretta} e si scrive $\sum_{i\in I}M_i=\Dirsum_{i\in I}M_i$.
\end{definition}


Siano $R_1,\ldots,R_n$ anelli, $R=R_1\times\ldots\times R_n$. Se $M_i$ è un $R_i$-modulo sinistro per $i=1,\ldots,n$, allora $M=M_1\times\ldots\times M_n$ ha una struttura di $R$-modulo sinistro, con la moltiplicazione data da $(a_1,\ldots,a_n)(x_1,\ldots,x_n)=(a_1x_1,\ldots,a_nx_n)$.

\begin{proposition}\thlabel{modules-over-product-of-rings}
Siano $R_1,\ldots,R_n$ anelli, $R=R_1\times\ldots\times R_n$, $M$ un $R$-modulo sinistro. Allora esistono $M_1,\ldots,M_n$ tali che $M_i$ è un $R_i$-modulo sinistro e $M\iso M_1\times\ldots\times M_n$.
\end{proposition}
\begin{proof}
Per $i=1,\ldots, n$ sia $\pi_i:R\to R_i$ la proiezione, $e_i=(0,\ldots,1_{R_i},\ldots,0)\in R$, $M_i=e_iM\le M$. Osserviamo che per ogni $a\in R\comma x\in M$ vale
$$
a(e_ix)=ae_i^2x=(ae_i)(e_ix)=\pi_i(a)(e_ix),
$$
pertanto $M_i$ è un $R_i$-modulo. Vale inoltre $M_i=\{x\in M:\forall j\neq i\quad e_jx=0\}$: l'inclusione $\subseteq$ segue da $e_ie_j=0$ per $i\neq j$; se invece $e_jx=0$ per ogni $j\neq i$, allora
$$
1_Rx=(e_1+\ldots+e_n)x=e_ix\in M_i.
$$
Consideriamo ora l'omomorfismo di $R$-moduli
\begin{align*}
\varphi:M&\longrightarrow M_1\times\ldots\times M_n\\
x&\longmapsto (e_1x,\ldots,e_nx)
\end{align*}
Verifichiamo che $\varphi$ è iniettivo: se $\varphi(x)=0$ allora
$$
x=1_Rx=(e_1+\ldots+e_n)x=0.
$$
Inoltre $\varphi$ è suriettivo: se $(x_1,\ldots,x_n)\in M_1\times\ldots\times M_n$ vale
$$
\varphi(x_1+\ldots+x_n)=(e_1x_1,\ldots,e_nx_n)=(x_1,\ldots,x_n).
$$
Dunque $\varphi$ è un isomorfismo.
\end{proof}
Manteniamo le notazioni della proposizione precedente; inoltre, dato un qualunque $R$-modulo sinistro $N$, indichiamo con $N_i\subseteq N$ il sottomodulo $e_iN$, in modo che $N\iso N_1\times\ldots\times N_n$.


\begin{corollary}\thlabel{homomorphisms-modules-over-product-of-rings}
Siano $M,N$ $R$-moduli sinistri, $f:M\to N$ un omomorfismo di moduli. Allora $f=f_1\times\ldots\times f_n$, dove $f_i=f|_{M_i}:M_i\to N_i$.
\end{corollary}
\begin{proof}
È sufficiente osservare che $f(e_ix)=e_if(x)\in N_i$.
\end{proof}

\begin{corollary}\thlabel{submodules-over-product-of-rings}
Se $N\le M$ è un sottomodulo, allora $N\iso N_1\times\ldots\times N_n$ con $N_i\le M_i$.
\end{corollary}
\begin{proof}
È sufficiente osservare che $N_i=e_iN\le e_iM=M_i$. 
\end{proof}


\section{Moduli Semplici e Semisemplici}

\begin{definition}
Sia $M$ un $R$-modulo sinistro. Si dice che $M$ è un \emph{modulo semplice} se è non nullo e ammette solo $0$ come sottomodulo proprio.
\end{definition}

\begin{proposition}\thlabel{simple-module-image}
Siano $M,N$ $R$-moduli sinistri, $f:M\to N$ un omomorfismo suriettivo di moduli. Se $M$ è semplice, allora $N$ è nullo oppure isomorfo a $M$ (e dunque semplice).
\end{proposition}
\begin{proof}
Vale $N\iso M/\ker f$. Ma $\ker f$ è nullo oppure è tutto $M$, dunque $N$ è nullo oppure isomorfo a $M$.
\end{proof}


\begin{definition}
Sia $M$ un $R$-modulo sinistro. Si dice che $M$ è \emph{finitamente generato} se esistono $x_1,\ldots x_n\in M$ tali che $M=Rx_1+\ldots+Rx_n$.
\end{definition}

\begin{definition}
Sia $A$ una $\mathbb{K}$-algebra non nulla, $M$ un $A$-modulo sinistro. Abbiamo visto che possiamo immaginare $\mathbb{K}$ immerso in $A$ (anzi, in $Z(A)$), dunque $M$ possiede una struttura di $\mathbb{K}$-spazio vettoriale. $M$ si dice \emph{finito} se ha dimensione finita come $\mathbb{K}$-spazio vettoriale.
\end{definition}

\begin{proposition}\thlabel{finite-module-equivalent}
Sia $A$ una $\mathbb{K}$-algebra finita non nulla, $M$ un $A$-modulo sinistro. Allora $M$ è finito se e solo se è finitamente generato.
\end{proposition}
\begin{proof}
\leavevmode
\begin{itemize}
\item[$(\Rightarrow)$] Supponiamo che $M$ sia finito. Allora esistono $x_1,\ldots, x_n\in M$ tali che $\mathbb{K}x_1+\ldots+\mathbb{K}x_n=M$. Ma $\mathbb{K}\subseteq A$, dunque $Ax_1+\ldots+Ax_n=M$, ovvero $M$ è finitamente generato.
\item[$(\Leftarrow)$] Supponiamo che $M$ sia finitamente generato. Allora esistono $x_1,\ldots x_n\in M$ tali che $Ax_1+\ldots+Ax_n=M$. Ma $A$ è finita, quindi esistono $a_1,\ldots a_m\in A$ tali che $\mathbb{K}a_1+\ldots+\mathbb{K}a_m=A$. Allora
$$
\sum_{i=1}^{n}\sum_{j=1}^{m}\mathbb{K}(a_jx_i)=M,
$$
dunque $M$ è finito.
\end{itemize}
\end{proof}



\begin{proposition}\thlabel{semisimple-module-equivalent}
Sia $A$ una $\mathbb{K}$-algebra finita non nulla, $M$ un $A$-modulo sinistro finito non nullo. Allora sono equivalenti:
\begin{enumerate}[(i)]
\item esiste una famiglia finita $\{M_i\}_{i\in I}$ di sottomoduli semplici di $M$ tali che $M=\sum_{i\in I}M_i$;
\item esiste una famiglia finita $\{M_i\}_{i\in I}$ di sottomoduli semplici di $M$ tali che $M=\Dirsum_{i\in I}M_i$;
\item per ogni sottomodulo $N\le M$ esiste un sottomodulo $N'\le M$ tale che $M=N\Dirsum N'$.
\end{enumerate}
\end{proposition}
\begin{proof}
\leavevmode
\begin{itemize}
\item[(i)$\Rightarrow$(iii)] Sia $J\subseteq I$ un sottoinsieme massimale tale che $N\cap M_j=0$ per ogni $j\in J$. Sia $N'=\sum_{j\in J}M_j\le M$. Supponiamo per assurdo che $N+N'\neq M$. Osserviamo che per ogni $i\in I$, essendo $M_i$ semplice, vale uno fra $M_i\cap(N+N')=0$ e $M_i\cap(N+N')=M_i$. Poiché $N+N'\neq M$, deve esistere un $i\in I$ tale che $M_i\cap(N+N')=0$. Ma allora $M_i\cap N'=0$ (quindi $i\not\in J$) e $M_i\cap N=0$, dunque $J$ non era massimale ($J\cup\{i\}$ è strettamente più grande e soddisfa la proprietà richiesta), assurdo.
\item[(iii)$\Rightarrow$(ii)] Se $M$ è semplice la tesi è banalmente vera. Altrimenti sia $N\le M$ un sottomodulo proprio non nullo. Per ipotesi esiste un sottomodulo $N'\le M$ tale che $M=N\dirsum N'$ (dunque $N'$ è proprio e non nullo). Naturalmente le dimensioni di $N$ e di $N$' come $\mathbb{K}$-spazi vettoriali sono minori di quella di $M$, dunque si può concludere per induzione.
\item[(ii)$\Rightarrow$(i)] È ovvio.
\end{itemize}
\end{proof}

\begin{definition}
Sia $A$ una $\mathbb{K}$-algebra non nulla, $M$ un $A$-modulo sinistro finito non nullo. $M$ si dice \emph{semisemplice} se soddisfa una delle proprietà equivalenti della \thref{semisimple-module-equivalent}.
\end{definition}

\begin{proposition}\thlabel{semisimple-modules}
Sia $A$ una $\mathbb{K}$-algebra finita non nulla, $M,N,L$ $A$-moduli sinistri.
\begin{enumerate}[(i)]
\item Se $M\le L$ e $N\le L$ sono semisemplici, allora $M+N$ è semisemplice.
\item Se $M$ e $N$ sono semisemplici, allora $M\dirsum N$ è semisemplice.
\item Se $f:M\to N$ è un omomorfismo suriettivo di moduli e $M$ è semisemplice, allora $N$ è semisemplice o nullo.
\item Se $N\le M$ e $M$ è semisemplice, allora $N$ e $M/N$ sono semisemplici o nulli.
\end{enumerate}
\end{proposition}
\begin{proof}
\leavevmode
\begin{enumerate}[(i)]
\item È ovvio dalla caratterizzazione (i) della \thref{semisimple-module-equivalent}.
\item È ovvio dalla caratterizzazione (ii) della \thref{semisimple-module-equivalent}.
\item Sia $M=\sum_{i\in I}M_i$ con gli $M_i$ semplici. Allora $N=f(M)=\sum_{i\in I}f(M_i)$, ma $f(M_i)$ è nullo oppure semplice per ogni $i\in I$ (\thref{simple-module-image}), dunque $N$ è semisemplice o nullo.
\item $M/N$ è l'immagine della proiezione al quoziente di $M$, dunque è semisemplice o nullo. Scriviamo ora $M=N\dirsum N'$ per un qualche $N'\le M$. Allora $N\iso M/N'$ (semplice verifica), dunque $N$ è un quoziente di $M$ ed è pertanto semisemplice o nullo.
\end{enumerate}
\end{proof}


\begin{definition}
Una $\mathbb{K}$-algebra finita non nulla $A$ si dice \emph{semisemplice a sinistra} se $\psub{A}{A}$ (ovvero $A$ vista come $A$-modulo sinistro) è semisemplice.
\end{definition}


\begin{proposition}\thlabel{modules-on-semisimple-algebra}
Sia A una $\mathbb{K}$-algebra semisemplice a sinistra. Allora ogni $A$-modulo sinistro finito non nullo è semisemplice.
\end{proposition}
\begin{proof}
Sia $M$ un $A$-modulo finito non nullo. Per la \thref{finite-module-equivalent} $M$ è finitamente generato; diciamo che $M=Ax_1+\ldots+Ax_n$. Consideriamo l'omomorfismo di moduli
\begin{align*}
\varphi:(\psub{A}{A})^n&\longrightarrow M\\
(a_1,\ldots,a_n)&\longmapsto a_1x_1+\ldots+a_nx_n
\end{align*}
È evidente che $\varphi$ è suriettivo, dunque $M\iso (\psub{A}{A})^n/\ker\varphi$. Ma $(\psub{A}{A})^n$ è semisemplice (è somma diretta di $n$ copie di $\psub{A}{A}$); per la \thref{semisimple-modules}, $M$ è semisemplice.
\end{proof}


\begin{proposition}\thlabel{simple-modules-on-semisimple-algebra}
Sia $A$ una $\mathbb{K}$-algebra semisemplice a sinistra, diciamo $\psub{A}{A}=\Dirsum_{i=1}^{n}M_i$ con gli $M_i$ semplici. Allora ogni $A$-modulo sinistro semplice è isomorfo a un qualche $M_i$.
\end{proposition}
\begin{proof}
Sia $M$ un $A$-modulo sinistro semplice. Consideriamo l'omomorfismo di moduli
\begin{align*}
\varphi:\psub{A}{A}&\longrightarrow M\\
a&\longmapsto a1_M
\end{align*}
$\varphi$ è non nullo ($\varphi(1_A)=1_M\neq 0$), dunque è suriettivo. Vale $M=\sum_{i=1}^{n}\varphi(M_i)$. $M$ è non nullo, pertanto per almeno un $i$ il sottomodulo $\varphi(M_i)\le M$ è non nullo, ma allora $\varphi(M_i)=M$, e per la \thref{simple-module-image} $M\iso M_i$.
\end{proof}


\section{Algebre di Divisione}

\begin{definition}
Sia $D$ una $\mathbb{K}$-algebra non nulla. Si dice che $D$ è un'\emph{algebra di divisione} se per ogni $a\in A\setminus\{0\}$ esiste $b\in A$ tale che $ab=ba=1$.
\end{definition}


Se $A$ è una $\mathbb{K}$-algebra e $M$ un $A$-modulo sinistro. Allora $M$ acquista una struttura di spazio vettoriale su $\mathbb{K}$, e lo spazio vettoriale $\End_\mathbb{K}(M)$ dotato dell'operazione di composizione è una $\mathbb{K}$-algebra. L'insieme $\End_A(M)$ degli endomorfismi di moduli da $M$ in $M$ è una sottoalgebra di $\End_\mathbb{K}(A)$.


\begin{proposition}\thlabel{module-schur-lemma}
Sia $A$ una $\mathbb{K}$-algebra non nulla, $M$ un $A$-modulo sinistro semplice. Allora $\End_A(M)$ è un'algebra di divisione.
\end{proposition}
\begin{proof}
Sia $\varphi\in\End_A(M)\setminus\{0\}$. Poiché $M$ è semplice e $\varphi$ è non nulla, $\im\varphi=M$ e $\ker\varphi=0$, dunque $\varphi$ è suriettiva e iniettiva, ovvero è invertibile.
\end{proof}


\begin{proposition}\thlabel{division-algebra-closed-fields}
Sia $D$ una $\mathbb{K}$-algebra di divisione finita. Supponiamo che $\mathbb{K}$ sia algebricamente chiuso. Allora $D=\mathbb{K}$.
\end{proposition}
\begin{proof}
Sia $a\in D\setminus\{0\}$. Consideriamo l'omomorfismo di spazi vettoriali finitamente generati $\varphi:D\to D$ definito da $\varphi(x)=ax$.  Allora $\varphi$ ammette un un autovettore $x\in D\setminus\{0\}$ per l'autovalore $\lambda\in\mathbb{K}$, ovvero $(a-\lambda)x=0$. Ma $x$ è non nullo e $D$ è un'algebra di divisione, pertanto $a=\lambda$. Abbiamo dimostrato che $D\subseteq\mathbb{K}$, dunque $D=\mathbb{K}$.
\end{proof}


\begin{proposition}\thlabel{division-algebras-over-R}
Sia $D$ una $\mathbb{R}$-algebra di divisione finita. Allora $D$ è isomorfa a uno fra $\mathbb{R}\comma\mathbb{C}\comma\mathbb{H}$.
\end{proposition}
\begin{proof}
Supponiamo per assurdo che la tesi sia falsa. Sappiamo che $\mathbb{R}\subsetneq D$. Sia allora $u\in D\setminus\mathbb{R}$, e sia $u\in\mathbb{R}[u]=\{p(u):p\in\mathbb{R}[x]\}$. $\mathbb{R}[u]$ è una sottoalgebra di $D$, ed è dunque finita. Preso un qualunque $v\in\mathbb{R}[u]$ esiste un $n\ge 0$ tale che i vettori $1,v,\ldots,v^n$ sono linearmente dipendenti, ovvero esiste un polinomio non nullo $p\in\mathbb{R}[x]$ tale che $p(v)=0$; prendiamo $p$ in modo che $p(0)=-1$. Vale dunque 
$$
a_nv^n+a_{n-1}v^{n-1}+\ldots+a_1v-1=0
$$
per opportuni $a_1,\ldots,a_n\in\mathbb{R}$, da cui
$$
v(a_nv^{n-1}+\ldots+a_1)=1,
$$
ovvero $v^{-1}\in\mathbb{R}[u]$. Osserviamo inoltre che $\mathbb{R}[u]$ è commutativo (le potenze di $u$ commutano fra loro e con gli elementi di $\mathbb{R}$), dunque $\mathbb{R}[u]$ è un campo. Poiché $\mathbb{R}[u]$ non è isomorfo a $\mathbb{R}$, deve necessariamente essere isomorfo a $\mathbb{C}$ (l'unica altra estensione finita di $\mathbb{R}$). Stiamo supponendo che $D$ non sia isomorfo a $\mathbb{C}$, dunque $\mathbb{R}[u]\subsetneq D$. Identifichiamo $\mathbb{C}$ con $\mathbb{R}[u]\subseteq D$. $\psub{\mathbb{C}}{D}$ (ovvero $D$ visto che $\mathbb{R}[u]$-modulo sinistro) è un $\mathbb{C}$-spazio vettoriale di dimensione finita. Consideriamo l'omomorfismo di spazi vettoriali $T:\psub{\mathbb{C}}{D}\to\psub{\mathbb{C}}{D}$ definito da $T(x)=xi$. Vale $T^2=-\id$, dunque per decomposizione primaria vale $\psub{\mathbb{C}}{D}=D^+\dirsum D^-$, dove $D^+=\{x\in D:xi=ix\}$ e $D^-=\{x\in D:xi=-ix\}$. Ovviamente $\mathbb{C}\subseteq D^+$; se per assurdo $\mathbb{C}\subsetneq D^+$, preso un $u'\in D^+\setminus\mathbb{C}$ e applicando un ragionamento analogo a quello di prima otterremmo che $\mathbb{C}[u']\subseteq D^+$ è un'estensione finita propria di $\mathbb{C}$, che è assurdo. Pertanto $D^+=\mathbb{C}$. Sia ora $v\in D^-$; consideriamo l'applicazione $\mathbb{R}$-lineare $\varphi:D\to D$ definita da $\varphi(x)=xv$. Poiché $v$ è invertibile, $\varphi$ è un isomorfismo. Osserviamo inoltre che $\varphi(D^+)\subseteq D^-$ e $\varphi(D^-)\subseteq D^+$, dunque valgono le uguaglianze e $D^-=D^+v$. Segue che $v^2\in D^+$, pertanto commuta con tutti gli elementi di $D^+$ (che è commutativo) e di $D^-$ (che è uguale a $D^+v$), dunque commuta con tutti gli elementi di $D$. Si può verificare facilmente che il centro di $D$ è $\mathbb{R}$: infatti contiene $\mathbb{R}$, è contenuto in $D^+$ (ogni elemento del centro deve commutare con $i$) e non è tutto $D^+$ ($i$ non sta nel centro, dato che $D^-$ è non vuoto), pertanto è esattamente $\mathbb{R}$. Allora $u^2\in\mathbb{R}$; inoltre $u^2<0$: infatti se fosse $u^2\ge 0$ esisterebbe $a\in\mathbb{R}$ tale che $u^2=a^2$, da cui $0=(u+a)(u-a)$, che è assurdo poiché $u\not\in\mathbb{R}$. Ponendo $j=(-u^2)^{-\frac{1}{2}}u$ e $k=ij$ si può verificare che $1,i,j,k$ generano $D$ come $\mathbb{R}$-spazio vettoriale e soddisfano gli assiomi dei quaternioni, dunque $D\iso\mathbb{H}$.
\end{proof}


