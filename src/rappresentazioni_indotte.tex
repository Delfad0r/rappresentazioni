\chapter{Rappresentazioni indotte}

\section{Definizione e proprietà}

\begin{definition}
Sia $G$ un gruppo, $H\le G$ un sottogruppo, $\Rep{\sigma}{H}{W}$ una rappresentazione. Si dice \emph{rappresentazione indotta} da $\sigma$ una rappresentazione $\Rep{\Ind_H^G\sigma}{G}{V}$ dotata di un'applicazione $\iota\in\Hom_H(W,V)$ che soddisfa la seguente proprietà universale: per ogni rappresentazione $\Rep{\tau}{G}{Z}$ e per ogni applicazione $f\in\Hom_H(W,Z)$ esiste un'unica applicazione $\bar{f}\in\Hom_G(V,Z)$ che fa commutare il diagramma
$$
\begin{diagram}
W\arrow{r}{f}\arrow[swap]{d}{\iota}&Z\\
V\arrow[swap]{ru}{\bar{f}}
\end{diagram}
$$
ovvero tale che $f=\bar{f}\circ\iota$.
\end{definition}

Se $G$ è un gruppo, $H\le G$ un sottogruppo, $\Rep{\rho}{G}{V}$ una rappresentazione, indichiamo con $\Res_H^G\rho$ la rappresentazione $\Rep{\rho|_H}{H}{V}$. Allora dalla definizione segue immediatamente
$$
\Hom(\Ind_H^G\sigma,\tau)\iso\Hom(\sigma,\Res_H^G\tau)
$$
dove l'isomorfismo è dato dalla mappa $(\bar{f}\mapsto\bar{f}\circ\iota)$.\\
Per snellire la notazione, scriveremo $\Ind\sigma$ in luogo di $\Ind_H^G\sigma$ e $\Res\rho$ in luogo di $\Res_H^G\rho$ quando non vi fossero rischi di ambiguità.

\begin{proposition}\thlabel{induced-representation-existence}
Sia $G$ un gruppo, $H\le G$ un sottogruppo, $\Rep{\sigma}{H}{W}$ una rappresentazione. Allora esiste una rappresentazione $\Rep{\Ind\sigma}{G}{V}$ (dotata di un'applicazione $\iota\in\Hom_H(W,V)$) che è indotta da $\sigma$.
\end{proposition}
\begin{proof}
Sia $\Rep{\mathcal{R}}{G}{\mathbb{K}[G]}$ la rappresentazione regolare di $G$, con $\{e_g\}_{g\in G}$ come base. Sia $\Rep{\triv}{G}{W}$ la rappresentazione banale di $G$ su $W$, e sia $\rho=\mathcal{R}\tensor\triv$ una rappresentazione su $\mathbb{K}[G]\tensor W$. Sia $j$ l'applicazione
\begin{alignat*}{2}
j:W&\longrightarrow&\mathbb{K}[G]&\tensor W\\
w&\longmapsto&e_1&\tensor w
\end{alignat*}
e sia
$$
K=\langle e_{gh}\tensor w-e_g\tensor\sigma(h)w:w\in W,g\in G,h\in H\rangle.
$$
Osserviamo che $K$ è un sottospazio $G$-stabile di $\mathbb{K}[G]\tensor W$: infatti per ogni $g,g'\in G\comma h\in H\comma w\in W$ vale
$$
(\mathcal{R}\tensor\triv)(g')(e_{gh}\tensor w-e_g\tensor\sigma(h)w)=e_{g'gh}\tensor w-e_{g'g}\tensor\sigma(h)w\in K.
$$
Pertanto (\thref{representation-quotient}) $\rho$ passa al quoziente $(\mathbb{K}[G]\tensor W)/K$; posto $V=(\mathbb{K}[G]\tensor W)/K$, sia $\pi:\mathbb{K}[G]\tensor W\to V$ la proiezione, $\Rep{\Ind\sigma}{G}{V}$ la rappresentazione corrispondente. Sia infine $\iota=\pi\circ j$, che si verifica facilmente essere un elemento di $\Hom_H(W,V)$: per ogni $h\in H\comma w\in W$ vale
\begin{align*}
\iota(\sigma(h)w)&=\pi(e_1\tensor\sigma(h)w)\\
&=\pi(e_h\tensor w)\\
&=(\Ind\sigma)(h)\pi(e_1\tensor w)\\
&=(\Ind\sigma)(h)\iota(w).
\end{align*}
Mostriamo ora che $\Ind\sigma$ (con applicazione associata $\iota$) è effettivamente indotta da $\sigma$. Sia $\Rep{\tau}{G}{Z}$ una rappresentazione, $f\in\Hom_H(W,Z)$ un'applicazione: dobbiamo trovare $\bar{f}\in\Hom_G(V,Z)$ tale che $f=\bar{f}\circ\iota$. Per la proprietà universale del prodotto tensore esiste un'unica applicazione lineare
\begin{alignat*}{2}
l:\mathbb{K}[G]&\tensor W&&\longrightarrow Z\\
e_g&\tensor w&&\longmapsto\tau(g)f(w)
\end{alignat*}
Osserviamo che $l\circ j=f$, e che $l$ è un omomorfismo di rappresentazioni: per ogni $g,g'\in G\comma w\in W$ vale
\begin{align*}
l((\mathcal{R}\tensor\triv)(g')(e_g\tensor w))&=l(e_{g'g}\tensor w)\\
&=\tau(g'g)f(w)\\
&=\tau(g')\tau(g)f(w)\\
&=\tau(g')l(e_g\tensor w).
\end{align*}
Inoltre si vede facilmente che $K\subseteq\ker l$: infatti per ogni $w\in W\comma g\in G\comma h\in H$ vale
\begin{align*}
l(e_{gh}\tensor w-e_g\tensor\sigma(h)w)&=\tau(gh)f(w)-\tau(g)f(\sigma(h)w)\\
&=\tau(gh)f(w)-\tau(g)\tau(h)f(w)\\
&=0,
\end{align*}
dove abbiamo usato che $f$ è un omomorfismo di rappresentazioni. Allora (\thref{representation-quotient-universal}) $l$ passa al quoziente, ovvero esiste un'unica applicazione $\bar{f}\in\Hom_G(V,Z)$ tale che $l=\bar{f}\circ\pi$. Vale inoltre $f=l\circ j=\bar{f}\circ\pi\circ j=\bar{f}\circ\iota$. Dunque $\bar{f}$ è l'omomorfismo cercato.
$$
\begin{diagram}
W\arrow{rr}{f}\arrow{rd}{j}\arrow[bend right=45,swap]{rrd}{\iota}&&Z\\
&\mathbb{K}[G]\tensor W\arrow{ru}{l}\arrow{r}{\pi}&V\arrow[swap]{u}{\bar{f}}
\end{diagram}
$$
Per mostrare l'unicità di $\bar{f}$, osserviamo preliminarmente che $l$ è l'unico omomorfismo di rappresentazioni da $\mathbb{K}[G]\tensor W$ in $Z$ tale che $l\circ j=f$: sia infatti $l'$ un omomorfismo che soddisfa questa proprietà, e siano $g\in G\comma w\in W$; vale allora
\begin{align*}
l'(e_g\tensor w)&=l'((\mathcal{R}\tensor \triv)(g)(e_1\tensor w))\\
&=\tau(g)l'(e_1\tensor w)\\
&=\tau(g)l'(j(w))\\
&=\tau(g)f(w).
\end{align*}
Sia ora $\bar{f}'\in\Hom_G(V,Z)$ tale che $\bar{f}'\circ\iota=f$. Vale dunque $f=(\bar{f}'\circ\pi)\circ j$; per l'unicità di $l$ abbiamo che $\bar{f}'\circ\pi=l$, e dall'unicità di $\bar{f}$ segue $\bar{f}'=\bar{f}$.
\end{proof}

\begin{proposition}\thlabel{induced-representation-uniqueness}
Sia $G$ un gruppo, $H\le G$ un sottogruppo, $\Rep{\sigma}{H}{W}$ una rappresentazione, $\Rep{\rho}{G}{V}\comma\Rep{\bar\rho}{G}{\bar{V}}$ rappresentazioni indotte da $\sigma$ con applicazioni associate rispettivamente $\iota\comma\bar\iota$. Allora $\rho$ e $\bar\rho$ sono isomorfe mediante un isomorfismo $\psi\in\Hom(\rho,\rho')$ tale che $\bar\iota=\psi\circ\iota$.
\end{proposition}
\begin{proof}
La dimostrazione è identica a quella della \thref{free-vector-space-uniqueness} e fa uso esclusivamente della proprietà universale della rappresentazione indotta.
\end{proof}

D'ora in poi parleremo dunque \emph{della} rappresentazione indotta da una rappresentazione, dato che abbiamo dimostrato che questa esiste ed è unica a meno di isomorfismo.

\begin{proposition}\thlabel{induced-representation-generated}
Sia $G$ un gruppo, $H\subseteq G$ un sottogruppo, $\Rep{\sigma}{H}{W}$ una rappresentazione, $\Rep{\rho}{G}{V}$ la rappresentazione indotta da $\sigma$ con applicazione associata $\iota\in\Hom_H(W,V)$. Allora
$$
V=\langle\rho(g)\iota(w):g\in G,w\in W\rangle
$$
\end{proposition}
\begin{proof}
Sia 
$$
Z=\langle\rho(g)\iota(w):g\in G,w\in W\rangle\subseteq V.
$$
Osserviamo che $Z$ è un sottospazio $G$-stabile, dunque $\rho$ passa al quoziente $V/Z$: sia $\Rep{\bar\rho}{G}{V/W}$ la rappresentazione quoziente, e sia $\pi:V\to V/W$ la proiezione. Consideriamo l'applicazione nulla $0|_W\in\Hom_H(W,V/W)$. È evidente che $0|_W=0|_V\circ\iota$. D'altro canto vale $\pi(\iota(w))=0$ per ogni $w\in W$, dunque $0|_W=\pi\circ\iota$. Segue che il diagramma
$$
\begin{diagram}
W\arrow{r}{0|_W}\arrow[swap]{d}{\iota}&V/Z\\
V\arrow{ru}{\pi}\arrow[bend right=15,swap]{ru}{0|_V}
\end{diagram}
$$
commuta, dunque $\pi=0|_V$, ovvero $Z=V$.
\end{proof}

\begin{proposition}\thlabel{induced-representation-basis}
Sia $G$ un gruppo, $H\le G$ un sottogruppo, $\Rep{\sigma}{H}{W}$ una rappresentazione, $\Rep{\rho}{G}{V}$ la rappresentazione indotta da $\sigma$ con applicazione associata $\iota$. Sia $\{w_i\}_{i\in I}$ una base di $W$, e sia $R\subseteq G$ un insieme di rappresentanti di $G/H$. Allora $\mathcal{B}=\{\rho(r)\iota(w_i)\}_{r\in R,i\in I}$ è una base di $V$.
\end{proposition}
\begin{proof}
Mostriamo innanzitutto che $\mathcal{B}$ è un insieme di generatori: se $g\in G$ e $w\in W$, allora esistono $h\in H\comma r\in R$ tali che $g=rh$, quindi 
$$
\rho(g)\iota(w)=\rho(r)\rho(h)\iota(w)=\rho(r)\iota(\sigma(h)w)\in\langle\mathcal{B}\rangle.
$$
Dalla \thref{induced-representation-generated} segue che $V=\langle B\rangle$. Mostriamo ora che gli elementi di $\mathcal{B}$ sono linearmente indipendenti. Sia $\Rep{\mathcal{R}}{G}{\mathbb{K}[G]}$ la rappresentazione regolare di $G$. Siano $a_{ri}\in\mathbb{K}$ coefficienti tali che
$$
0=\sum_{r\in R}\sum_{i\in I}a_{ri}\rho(r)\iota(w_i).
$$
Fissiamo un $j\in I$; sia $f_j$ l'applicazione lineare
\begin{align*}
f_j:W&\longrightarrow\mathbb{K}[G]\\
w&\longmapsto\sum_{h\in H}w_j^*(\sigma(h^{-1})w)e_h
\end{align*}
Osserviamo che $f_j\in\Hom_H(W,\mathbb{K}[G])$: per ogni $h'\in H\comma w\in W$ vale
\begin{align*}
\mathcal{R}(h)f_j(w)&=\sum_{h\in H}w_j^*(\sigma(h^{-1})w)\mathcal{R}(h')e_h\\
&=\sum_{h\in H}w_j^*(\sigma(h'h)^{-1}\sigma(h')w)e_{h'h}\\
&=\sum_{h\in H}w_j^*(\sigma(h)\sigma(h')w)e_h\\
&=f_j(\sigma(h')w).
\end{align*}
Per la proprietà universale della rappresentazione indotta esiste un'applicazione $\bar{f}_j\in\Hom_G(V,\mathbb{K}[G])$ tale che $f_j=\bar{f}_j\circ\iota$. Allora vale
\begin{align*}
0&=\sum_{r\in R}\sum_{i\in I}a_{ri}\bar{f}_j(\rho(r)\iota(w_i))\\
&=\sum_{r\in R}\sum_{i\in I}a_{ri}\mathcal{R}(r)\bar{f}_j(\iota(w_i))\\
&=\sum_{r\in R}\sum_{i\in I}a_{ri}\mathcal{R}(r)f_j(w_i)\\
&=\sum_{r\in R}\sum_{i\in I}a_{ri}\sum_{h\in H}w_j^*(\sigma(h^{-1})w_i)\mathcal{R}(r)e_h\\
&=\sum_{r\in R}\sum_{h\in H}\biggl(\sum_{i\in I}a_{ri}w_j^*(\sigma(h^{-1})w_i)\biggr)e_{rh}.
\end{align*}
Ma $\{e_{rh}\}_{r\in R,h\in H}$ è una base di $\mathbb{K}[G]$, quindi per ogni $r\in R\comma h\in H$ vale
$$
0=\sum_{i\in I}a_{ri}w_j^*(\sigma(h^{-1})w_i);
$$
scegliendo in particolare $h=1$ otteniamo
$$
0=\sum_{i\in I}a_{ri}w_j^*(\sigma(1)w_i)=a_{rj}.
$$
Applicando questo ragionamento per ogni $j\in I$ risulta che tutti i coefficienti sono nulli, dunque gli elementi di $\mathcal{B}$ sono linearmente indipendenti.
\end{proof}

\begin{corollary}\thlabel{induced-representation-dimension}
Sia $G$ un gruppo, $H\le G$ un sottogruppo, $\Rep{\sigma}{H}{W}$ una rappresentazione. Allora
$$
\deg\Ind\sigma=[G:H]\cdot\deg\sigma
$$
\end{corollary}

La \thref{induced-representation-basis} implica che $\iota$ è iniettiva: potremo quindi assumere, senza perdere di generalità, che $W$ sia un sottospazio di $V$ (e che $\iota$ sia l'identità).

\begin{proposition}\thlabel{induced-representation-dirsum}
Sia $G$ un gruppo, $H\le G$ un sottogruppo di indice finito, $R\subseteq G$ un insieme di rappresentanti di $G/H$. Sia $\Rep{\rho}{G}{V}$ una rappresentazione, $W\subseteq V$ un sottospazio $H$-stabile, $\sigma=\Res\rho|^W$. Allora sono equivalenti:
\begin{enumerate}[(i)]
\item $\rho$ è indotta da $\sigma$;
\item $V=\Dirsum_{r\in R}\rho(r)W$.
\end{enumerate}
\end{proposition}
\begin{proof}
L'implicazione (i)$\implies$(ii) segue dalla \thref{induced-representation-basis}. Per dimostrare l'altra consideriamo una rappresentazione $\Rep{\tau}{G}{Z}$ e un'applicazione $f\in\Hom_H(W,Z)$. Dobbiamo trovare $\bar{f}\in\Hom_G(V,Z)$ tale che $\bar{f}|_W=f$. Una tale $\bar{f}$, se esiste, deve soddisfare, per ogni $r\in R$, $\bar{f}\circ\rho(r)=\tau(r)\circ\bar{f}$, ovvero $\bar{f}=\tau(r)\circ\bar{f}\circ\rho(r)^{-1}$, dunque 
$$
\bar{f}|_{\rho(r)W}=\tau(r)\circ\bar{f}\circ\rho(r)^{-1}=\tau(r)\circ f\circ\rho(r)^{-1}.
$$
Segue che $\bar{f}$ è al più unica. Sia ora $\bar{f}$ l'applicazione lineare definita da $\bar{f}|_{\rho(r)W}=\tau(r)\circ f\circ\rho(r)^{-1}$ per ogni $r\in R$. Mostriamo che $\bar{f}\in\Hom_G(V,Z)$. Sia $v\in V\comma g\in G$; siano $r\in R\comma w\in W$ tali che $v=\rho(r)w$, e siano $r'\in R\comma h\in H$ tali che $gr=r'h$; vale allora
\begin{align*}
\bar{f}(\rho(g)v)&=\bar{f}(\rho(g)\rho(r)w)\\
&=\bar{f}(\rho(r')\rho(h)w)\\
&=\bar{f}|_{\rho(r')W}(\rho(r')\sigma(h)w)\\
&=\tau(r')f(\sigma(h)w)\\
&=\tau(r')\tau(h)f(w)\\
&=\tau(g)\tau(r)f(\rho(r)^{-1}v)\\
&=\tau(g)\bar{f}(v).
\end{align*}
Inoltre, se $r\in R\comma h\in H$ sono tali che $1=rh$ vale
$$
\bar{f}(w)=\bar{f}(\rho(r)\rho(h)w)=\tau(r)f(\sigma(h)w)=\tau(r)\tau(h)f(w)=f(w),
$$
ovvero $\bar{f}|_W=f$. Dunque $\bar{f}$ è l'omomorfismo cercato.
\end{proof}

\section{Funzioni Indotte}
Da qui alla fine del capitolo ci limiteremo a considerare (come già avevamo fatto con la teoria dei caratteri) rappresentazioni complesse di grado finito su gruppi finiti.

\begin{definition}
Sia $G$ un gruppo, $H\le G$ un sottogruppo, $f\in\Fun{H}{\mathbb{C}}$. Si definisce \emph{funzione indotta} da $f$ la funzione $\Ind_H^Gf\in\Fun{G}{\mathbb{C}}$ definita da
$$
(\Ind_H^Gf)(g)=\frac{1}{|H|}\sum_{\substack{s\in G\\s^{-1}gs\in H}}f(s^{-1}gs)
$$
\end{definition}

Se $G$ è un gruppo, $H\le G$ un sottogruppo, $k\in\Fun{G}{\mathbb{C}}$ una funzione, indichiamo con $\Res_H^Gk$ la funzione $k|_H\in\Fun{H}{\mathbb{C}}$.\\
Per snellire la notazione, scriveremo $\Ind f$ in luogo di $\Ind_H^G f$ e $\Res k$ in luogo di $\Res_H^G k$ quando non vi fossero rischi di ambiguità.\\
È evidente che, se $\Rep{\rho}{G}{V}$ è una rappresentazione, vale $\Res\chi_\rho=\chi_{\Res\rho}$. Vale una proprietà analoga per le rappresentazioni indotte.

\begin{proposition}\thlabel{induced-representation-character}
Sia $G$ un gruppo, $H\le G$ un sottogruppo, $\Rep{\sigma}{H}{W}$ una rappresentazione. Allora $\chi_{\Ind\sigma}=\Ind\chi_\sigma$.
\end{proposition}
\begin{proof}
Sia $\Rep{\rho=\Ind\sigma}{G}{V}$, e sia $R$ un insieme di rappresentanti di $G/H$. Per la \thref{induced-representation-dirsum} possiamo supporre $V=\Dirsum_{r\in R}\rho(r)W$. Osserviamo che, essendo $W$ $H$-stabile, , $\rho(g)W$ dipende solo dalla classe di $g$ in $G/H$; in particolare, $\rho(g)W=W$ se e solo se $g\in H$. Allora per ogni $g\in G$ vale
\begin{align*}
\chi_\rho(g)&=\tr\rho(g)\\
&=\sum_{\substack{r\in R\\\rho(gr)W=\rho(r)W}}\tr\rho(g)|_{\rho(r)W}\\
&=\sum_{\substack{r\in R\\\rho(r^{-1}gr)W=W}}\tr\rho(g)|_{\rho(r)W}\\
&=\sum_{\substack{r\in R\\r^{-1}gr\in H}}\tr\rho(g)|_{\rho(r)W}\\
&=\sum_{\substack{r\in R\\r^{-1}gr\in H}}\tr\left(\rho(r)^{-1}|_{\rho(r)W}\circ\rho(g)|_{\rho(r)W}\circ\rho(r)|_W\right)\\
&=\sum_{\substack{r\in R\\r^{-1}gr\in H}}\tr\rho(r^{-1}gr)|_W\\
&=\sum_{\substack{r\in R\\r^{-1}gr\in H}}\chi_\sigma(r^{-1}gr)
\end{align*}
dove abbiamo utilizzato la proprietà $\tr(f)=\tr(h^{-1}\circ f\circ h)$. Sia ora $s\in G$; $s$ si scrive in modo unico come $s=rh$ con $r\in R\comma h\in H$. Osserviamo che $s^{-1}gs=h^{-1}r^{-1}grh$, quindi $s^{-1}gs\in H$ se e solo se $r^{-1}gr\in H$, e in tal caso $\chi_\sigma(s^{-1}gs)=\chi_\sigma(r^{-1}gr)$ (il carattere è una funzione di classe). Segue che
\begin{align*}
\Ind\chi_\sigma(g)&=\frac{1}{|H|}\sum_{\substack{s\in S\\s^{-1}gs\in H}}\chi_\sigma(s^{-1}gs)\\
&=\frac{1}{|H|}\sum_{h\in H}\sum_{\substack{r\in R\\r^{-1}gr\in H}}\chi_\sigma(r^{-1}gr)\\
&=\sum_{\substack{r\in R\\r^{-1}gr\in H}}\chi_\sigma(r^{-1}gr)\\
&=\chi_\rho(g)
\end{align*}

\end{proof}

