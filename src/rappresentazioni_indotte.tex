\chapter{Rappresentazioni indotte}

\section{Definizione e proprietà}

\begin{definition}
Sia $G$ un gruppo, $H\le G$ un sottogruppo, $\Rep{\sigma}{H}{W}$ una rappresentazione. Si dice \emph{rappresentazione indotta} da $\sigma$ una rappresentazione $\Rep{\Ind_H^G\sigma}{G}{V}$ dotata di un'applicazione $\iota\in\Hom_H(W,V)$ che soddisfa la seguente proprietà universale: per ogni rappresentazione $\Rep{\tau}{G}{Z}$ e per ogni applicazione $f\in\Hom_H(W,Z)$ esiste un'unica applicazione $\bar{f}\in\Hom_G(V,Z)$ che fa commutare il diagramma
$$
\begin{diagram}
W\arrow{r}{f}\arrow[swap]{d}{\iota}&Z\\
V\arrow[swap]{ru}{\bar{f}}
\end{diagram}
$$
ovvero tale che $f=\bar{f}\circ\iota$.
\end{definition}

Per snellire la notazione, scriveremo $\Ind\sigma$ in luogo di $\Ind_H^G\sigma$ quando non vi fossero rischi di ambiguità.

\begin{proposition}\thlabel{induced-representation-existence}
Sia $G$ un gruppo, $H\le G$ un sottogruppo, $\Rep{\sigma}{H}{W}$ una rappresentazione. Allora esiste una rappresentazione $\Rep{\Ind\sigma}{G}{V}$ (dotata di un'applicazione $\iota\in\Hom_H(W,V)$) che è indotta da $\sigma$.
\end{proposition}
\begin{proof}
Sia $\Rep{\mathcal{R}}{G}{\mathbb{K}[G]}$ la rappresentazione regolare di $G$, con $\{e_g\}_{g\in G}$ come base. Sia $\Rep{\triv}{G}{W}$ la rappresentazione banale di $G$ su $W$, e sia $\rho=\mathcal{R}\tensor\triv$ una rappresentazione su $\mathbb{K}[G]\tensor W$. Sia
$$
K=\langle e_{gh}\tensor w-e_g\tensor\sigma(h)w:w\in W,g\in G,h\in H\rangle.
$$
Osserviamo che $K$ è un sottospazio $G$-stabile di $\mathbb{K}[G]$: infatti per ogni $g,g'\in G\comma h\in H\comma w\in W$ vale
$$
(\mathcal{R}\tensor\triv)(g')(e_{gh}\tensor w-e_g\tensor\sigma(h)w)=e_{g'gh}\tensor w-e_{g'g}\tensor\sigma(h)w\in K.
$$
Pertanto $\rho$ passa al quoziente $(\mathbb{K}[G]\tensor W)/K$; posto $V=(\mathbb{K}[G]\tensor W)/K$, sia $\pi:\mathbb{K}[G]\tensor W\to V$ la proiezione, $\Rep{\Ind\sigma}{G}{V}$ la rappresentazione corrispondente. Sia infine $\iota$ l'applicazione lineare
\begin{align*}
\iota:W&\longrightarrow V\\
w&\longmapsto \pi(e_1\tensor w)
\end{align*}
che si verifica facilmente essere un elemento di $\Hom_H(W,V)$: per ogni $h\in H\comma w\in W$ vale
\begin{align*}
\iota(\sigma(h)w)&=\pi(e_1\tensor\sigma(h)w)\\
&=\pi(e_h\tensor w)\\
&=(\Ind\sigma)(h)\pi(e_1\tensor w)\\
&=(\Ind\sigma)(h)\iota(w).
\end{align*}
Mostriamo ora che $\Ind\sigma$ (con applicazione associata $\iota$) è effettivamente indotta da $\sigma$. Sia $\Rep{\tau}{G}{Z}$ una rappresentazione, $f\in\Hom_H(W,Z)$ un'applicazione: dobbiamo trovare $\bar{f}\in\Hom_G(V,Z)$ tale che $f=\bar{f}\circ\iota$. Per la proprietà universale del prodotto tensore esiste un'unica applicazione lineare
\begin{alignat*}{2}
l:\mathbb{K}[G]&\tensor W&&\longrightarrow Z\\
e_g&\tensor w&&\longmapsto\tau(g)f(w)
\end{alignat*}
Si vede facilmente che $K\subseteq\ker l$: infatti per ogni $w\in W\comma g\in G\comma h\in H$ vale
\begin{align*}
l(e_{gh}\tensor w-e_g\tensor\sigma(h)w)&=\tau(gh)f(w)-\tau(g)f(\sigma(h)w)\\
&=\tau(gh)f(w)-\tau(g)\tau(h)f(w)\\
&=0,
\end{align*}
dove abbiamo usato che $f$ è un omomorfismo di rappresentazioni. Allora $l$ passa al quoziente, ovvero esiste un'unica applicazione lineare $\bar{f}:V\to Z$ tale che $l=\bar{f}\circ\pi$. Verifichiamo che $\bar{f}\in\Hom_G(V,Z)$: per ogni $g,g'\in G\comma w\in W$ vale
\begin{align*}
\bar{f}((\Ind\sigma)(g')\pi(e_g\tensor w))&=\bar{f}(\pi(e_{g'g}\tensor w))\\
&=l(e_{g'g}\tensor w)\\
&=\tau(g'g)f(w)\\
&=\tau(g')\tau(g)f(w)\\
&=\tau(g')l(e_g\tensor w)\\
&=\tau(g')\bar{f}(\pi(e_g\tensor w)).
\end{align*}
Verifichiamo infine che $f=\bar{f}\circ\iota$: dato $w\in W$ vale
$$
f(w)=\tau(1)f(w)=l(e_1\tensor w)=\bar{f}(\pi(e_1\tensor w))=\bar{f}(\iota(w)).
$$
Dunque $\Ind\sigma$ è la rappresentazione cercata.
$$
\begin{diagram}
W\arrow{rr}{f}\arrow{rd}{\tensor}\arrow[bend right=45,swap]{rrd}{\iota}&&Z\\
&\mathbb{K}[G]\tensor W\arrow{ru}{l}\arrow{r}{\pi}&V\arrow[swap]{u}{\bar{f}}
\end{diagram}
$$
%TODO: sistemare il diagramma (togliendo $\tensor$), fare l'unicità, forse fare un intero capitoletto sulle rappresentazioni quoziente?
\end{proof}
