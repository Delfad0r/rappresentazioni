\chapter{Introduzione alle Rappresentazioni}

\section{Rappresentazioni e Omomorfismi}

\begin{definition}
Sia $G$ un gruppo, $V$ uno spazio vettoriale. Si dice \emph{rappresentazione} (lineare) un omomorfismo di gruppi $\Rep{\rho}{G}{V}$. $\rho$ si dice \emph{rappresentazione complessa} (rispettivamente \emph{reale}) se $V$ è uno spazio vettoriale complesso (rispettivamente reale).
\end{definition}

Con lieve abuso di notazione, potremo chiamare $V$ stesso \emph{rappresentazione} quando sarà chiaro dal contesto quale rappresentazione stia agendo su $V$.

Poiché ogni rappresentazione $\Rep{\rho}{G}{V}$ è un'azione di $G$ su $V$, utilizzeremo spesso la notazione (usuale per le azioni di gruppi) $gv$ per indicare $\rho(g)v$.

\begin{definition}
Sia $\Rep{\rho}{G}{V}$ una rappresentazione. Si chiama \emph{grado} di $\rho$ (e si indica con $\deg\rho$) il numero $\dim V$.
\end{definition}

\begin{example}
Per ogni $n\in\mathbb{N}$ la rappresentazione $\Rep{n_G}{G}{\mathbb{K}^n}$ (che indicheremo con $n$ qualora non ci fossero rischi di ambiguità) che a ogni $g\in G$ associa l'identità è detta \emph{rappresentazione banale di grado $n$}. $0_G$ prende il nome di \emph{rappresentazione nulla}.
\end{example}

\begin{definition}
Siano $\Rep{\rho}{G}{V}\comma\Rep{\sigma}{G}{W}$ rappresentazioni. Un'applicazione lineare $\varphi:V\to W$ si dice \emph{omomorfismo di rappresentazioni} se per ogni $g\in G$ vale $\varphi\circ\rho(g)=\sigma(g)\circ\varphi$. L'insieme degli omomorfismi di rappresentazioni da $\rho$ in $\sigma$ si indica con $\Hom(\rho,\sigma)$ o $\Hom_G(V, W)$.
\end{definition}

\begin{definition}
Siano $\Rep{\rho}{G}{V}\comma\Rep{\sigma}{G}{W}$ rappresentazioni. Un'applicazione lineare $\varphi:V\to W$ si dice \emph{isomorfismo di rappresentazioni} se è un isomorfismo e un omomorfismo di rappresentazioni. Diciamo che $\rho$ e $\sigma$ sono \emph{isomorfe} (e scriviamo $\rho\isor\sigma$ o, più liberamente, $\rho=\sigma$) se esiste un isomorfismo di rappresentazioni da $\rho$ in $\sigma$.
\end{definition}

Utilizzeremo il simbolo $\isor$ per indicare l'isomorfismo anche quando confonderemo le rappresentazioni con i rispettivi spazi vettoriali: $V\isor W$ indica che $V$ e $W$ sono isomorfi come rappresentazioni, mentre $V\iso W$ indica che $V$ e $W$ sono isomorfi solo come spazi vettoriali.

\begin{proposition}\thlabel{representation-homomorphisms-properties}
Siano $\Rep{\rho}{G}{V}\comma\Rep{\sigma}{G}{W}\comma\Rep{\tau}{G}{U}$ rappresentazioni, $\varphi\in\Hom_G(V,W)\comma\psi\in\Hom_G(W,U)$.
\begin{enumerate}[(i)]
\item $\Hom_G(V,W)$ è un sottospazio vettoriale di $\Hom(V,W)$
\item $\psi\varphi\in\Hom_G(V,U)$
\item L'isomorfismo fra rappresentazioni è una relazione di equivalenza.
\end{enumerate}
\end{proposition}
\begin{proof}
Si tratta di semplici verifiche.
\end{proof}

\section{Somma di Rappresentazioni}

\begin{definition}
Siano $\Rep{\rho}{G}{V}\comma\Rep{\sigma}{G}{W}$ rappresentazioni. Si definisce \emph{somma} di $\rho$ e $\sigma$ la rappresentazione
\begin{align*}
\rho+\sigma G&\longrightarrow GL(V\dirsum W)\\
g&\longmapsto\rho(g)\dirsum\sigma(g)
\end{align*}
\end{definition}

Scriveremo $n\rho$ (dove $n\in\mathbb{N}$) per indicare la rappresentazione $\underbrace{\rho+\ldots+\rho}_{\text{$n$ volte}}$; in particolare, $0\rho=0_G$.

\begin{proposition}\thlabel{representation-direct-sum-properties}
Siano $\Rep{\rho}{G}{V}\comma\Rep{\sigma}{G}{W}\comma\Rep{\tau}{G}{U}$ rappresentazioni.
\begin{enumerate}[(i)]
\item $\deg(\rho+\sigma)=\deg\rho+\deg\sigma$
\item $\rho+\sigma\isor\sigma+\rho$
\item $(\rho+\sigma)+\tau\isor\rho+(\sigma+\tau)$
\end{enumerate}
\end{proposition}
\begin{proof}
Dimostriamo solo la (ii)
\begin{enumerate}[(i)]
\addtocounter{enumi}{1}
\item Sia $\psi:V\dirsum W\to W\dirsum V$ tale che $\psi(v\dirsum w)=w\dirsum v$ per ogni $v\in V\comma w\in W$. $\psi$ è ovviamente un isomorfismo di spazi vettoriali; mostriamo che è anche un omomorfismo di rappresentazioni.
\begin{align*}
\psi((\rho+\sigma)(g)(v\dirsum w))&=\psi(\rho(g)v\dirsum\sigma(g)w)\\
&=\sigma(g)w\dirsum\rho(g)v\\
&=(\sigma+\rho)(g)(w\dirsum v)\\
&=(\sigma+\rho)(g)\psi(v\dirsum w)
\end{align*}
\end{enumerate}
Molte dimostrazioni di proprietà delle operazioni fra rappresentazione sono analoghe a questa e verranno pertanto omesse.
\end{proof}

\begin{proposition}\thlabel{representation-projection-inclusion-homomorphism}
Siano $\Rep{\rho}{G}{V}\comma\Rep{\sigma}{G}{W}$ rappresentazioni, $\pi_V:V\dirsum W\to V$ la proiezione, $\iota_V:V\to V\dirsum W$ l'inclusione. Allora $\pi_V$ e $\iota_V$ sono omomorfismi di rappresentazioni.
\end{proposition}
\begin{proof}
Le verifiche sono semplici:
\begin{multline*}
\pi_V((\rho+\sigma)(g)(v\dirsum w))=\pi_V(\rho(g)v\dirsum\sigma(g)w)=\rho(g)v=\rho(g)\pi_V(v\dirsum w)\\
\iota_V(\rho(g)v)=\rho(g)v\dirsum 0=(\rho+\sigma)(g)(v\dirsum 0)=(\rho+\sigma)(g)\iota_V(v)\\
\end{multline*}
\end{proof}

\begin{proposition}\thlabel{representation-sum-homomorphisms}
Siano $\Rep{\rho}{G}{V}\comma\Rep{\sigma}{G}{W}\comma\Rep{\tau}{G}{U}$ rappresentazioni. Allora
\begin{align*}
\Hom(\rho+\sigma,\tau)\iso\Hom(\rho,\tau)\dirsum\Hom(\sigma,\tau)\\
\Hom(\rho,\sigma+\tau)\iso\Hom(\rho,\sigma)\dirsum\Hom(\rho,\tau)
\end{align*}
\end{proposition}
\begin{proof}
Dimostriamo solo la prima affermazione.\\
Siano $\pi_V:V\dirsum W\to V\comma\pi_W:V\dirsum W\to W$ le proiezioni, $\iota_V:V\to V\dirsum W\comma\iota_W:W\to V\dirsum W$ le inclusioni. Consideriamo l'applicazione lineare
\begin{alignat*}{2}
\Theta:\Hom_G(V,U)&\dirsum\Hom_G(W,U)&&\longrightarrow\Hom_G(V\dirsum W,U)\\
\varphi&\dirsum\psi&&\longmapsto\varphi\pi_V+\psi\pi_W
\end{alignat*}
Osserviamo che $\Theta$ è ben definita (cioè l'immagine sta davvero in $\Hom_G(V\dirsum W,U)$): infatti per la \thref{representation-projection-inclusion-homomorphism} $\pi_V\in\Hom_G(V\dirsum W,V)$, quindi $\varphi\pi_W\in\Hom_G(V\dirsum W,U)$. Per mostrare che $\Theta$ è un isomorfismo basta notare che l'inversa $\Theta^{-1}$ è data da
$$
\Theta^{-1}(\varphi)=\varphi\iota_V\dirsum\varphi\iota_W
$$
: infatti $\Theta^{-1}$ è ben definita per le stesse ragioni di sopra, e si verifica facilmente che $\Theta^{-1}\circ\Theta=\id\comma\Theta\circ\Theta^{-1}=\id$.
\end{proof}

\section{Sottorappresentazioni e Decomposizioni}

\begin{definition}
Sia $\Rep{\rho}{G}{V}$ una rappresentazione. Un sottospazio $W\subseteq V$ si dice \emph{$G$-stabile} se, per ogni $g\in G$, $W$ è $\rho(g)$-stabile, ovvero $\rho(g)(W)\subseteq W$.
\end{definition}

\begin{definition}
Sia $\Rep{\rho}{G}{V}$ una rappresentazione, $W\subseteq V$ un sottospazio $G$-stabile. Indichiamo con $\Rep{\rho|^W}{G}{W}$ la rappresentazione tale che $\rho|^W(g)=\rho(g)|_W$ per ogni $g\in G$. Diciamo che $\rho|^W$ è una \emph{sottorappresentazione} di $\rho$; se $W\neq\langle0\rangle$ e $W\neq V$ diciamo che $\rho|^W$ è una \emph{sottorappresentazione propria} di $\rho$.
\end{definition}

Coerentemente con il già citato abuso di notazione potremo dire che $W$ è una sottorappresentazione di $V$ se $W$ è $G$-stabile (invece del più proprio ``$\rho|^W$ è una sottorappresentazione di $\rho$'').

\begin{example}
Se $\Rep{\rho}{G}{V}\comma\Rep{\sigma}{G}{W}$ sono rappresentazioni, allora $\rho$ e $\sigma$ sono sottorappresentazioni di $\rho+\sigma$ (o $V$ e $W$ sono sottorappresentazioni di $V\dirsum W$).
\end{example}

\begin{proposition}\thlabel{subrepresentation-ker-im}
Siano $\Rep{\rho}{G}{V}\comma\Rep{\sigma}{G}{W}$ rappresentazioni, $\varphi\in\Hom(\rho,\sigma)$. Allora $\ker\varphi$ e $\im\varphi$ sono sottorappresentazioni, rispettivamente, di $V$ e di $W$.
\end{proposition}
\begin{proof}
Sia $v\in\ker\varphi$. Allora $\varphi(\rho(g)v)=\sigma(g)\varphi(v)=0$, quindi $\rho(g)v\in\ker\varphi$. Analogamente se $w\in\im\varphi$, allora $w=\varphi(v)$ per un qualche $v\in V$; segue che $\sigma(g)w=\sigma(g)\varphi(v)=\varphi(\rho(g)v)$, quindi $\sigma(g)w\in\im\varphi$.
\end{proof}

\begin{definition}
Una rappresentazione non nulla si dice \emph{irriducibile} se non ha sottorappresentazioni proprie.
\end{definition}

\begin{example}
Ogni rappresentazione di grado 1 è irriducibile.
\end{example}

Denotiamo con $\Irr(G)$ un insieme di rappresentanti delle classi di isomorfismo delle rappresentazioni irriducibili di $G$.

\begin{definition}
Una rappresentazione $\Rep{\rho}{G}{V}$ non nulla si dice \emph{indecomponibile} se non esistono sottorappresentazioni proprie $W\comma W'\subseteq V$ tali che $V=W\dirsum W'$.
\end{definition}

\begin{remark}
Ogni rappresentazione irriducibile è indecomponibile, mentre il viceversa è falso in generale.
\end{remark}

\begin{definition}
Una rappresentazione $\Rep{\rho}{G}{V}$ si dice \emph{completamente riducibile} se si scrive come somma di rappresentazioni irriducibili.
\end{definition}

\begin{example}
Ogni rappresentazione irriducibile è completamente riducibile.
\end{example}

\begin{proposition}\thlabel{representation-irreducible-finite-degree}
Sia $G$ un gruppo finito, $\Rep{\rho}{G}{V}$ una rappresentazione irriducibile. Allora $\deg\rho<\infty$.
\end{proposition}
\begin{proof}
Sia $v\in V$ un vettore non nullo. Sia $W=\langle\rho(g)v:g\in G\rangle\neq\langle 0\rangle$. È facile verificare che $W$ è $G$-stabile (basta farlo sui generatori), quindi (essendo $V$ irriducibile) $W=V$. Ma $\dim W\le|G|<\infty$.
\end{proof}

\begin{proposition}\thlabel{representation-finite-group-diagonalisable}
Sia $G$ è un gruppo finito di ordine $n$, $\Rep{\rho}{G}{V}$ una rappresentazione. Supponiamo che $\mathbb{K}$ sia algebricamente chiuso. Allora per ogni $g\in G$ gli autovalori di $\rho(g)$ sono radici $n$-esime dell'unità. Se inoltre supponiamo che $\ch\mathbb{K}\nmid n$, allora $\rho(g)$ è diagonalizzabile.
\end{proposition}
\begin{proof}
Sia $g\in G$. Osserviamo che $\rho(g)^n=\id$, dunque gli autovalori di $\rho(g)$ sono necessariamente radici $n$-esime dell'unità. Inoltre il polinomio minimo di $\rho(g)$ divide $x^n-1$. Supponiamo ora che $\ch\mathbb{K}\nmid n$; allora applicando il criterio della derivata si ottiene che $x^n-1$ non ha radici doppie, quindi il polinomio minimo di $\rho(g)$ è prodotto di fattori di primo grado. Per decomposizione primaria segue che $\rho(g)$ è diagonalizzabile.
\end{proof}

\begin{proposition}\thlabel{representation-finite-abelian-group}
Sia $G$ un gruppo abeliano finito, $\Rep{\rho}{G}{V}$ una rappresentazione di grado finito. Supponiamo che $\mathbb{K}$ sia algebricamente chiuso e che $\ch\mathbb{K}\nmid |G|$. Allora $V$ si decompone come somma di rappresentazioni di grado 1.
\end{proposition}
\begin{proof}
Sappiamo dalla \thref{representation-finite-group-diagonalisable} che ogni $\rho(g)$ è diagonalizzabile. Inoltre, poiché $G$ è abeliano, $\rho(G)$ è un sottogruppo abeliano di $GL(V)$, dunque gli elementi di $\rho(G)$ sono simultaneamente diagonalizzabili, ovvero esiste una base di V  di vettori $\{v_i\}_{i\in I}$ che sono autovettori per ogni $\rho(g)\in\rho(G)$. Ma allora $\langle v_i\rangle$ è un sottospazio $G$-stabile per ogni $i\in I$, quindi
$$
V=\Dirsum_{i\in I}\langle v_i\rangle
$$
come somma di rappresentazioni.
\end{proof}

\begin{proposition}\thlabel{subrepresentation-stable-complement}
Sia $G$ un gruppo finito, $\Rep{\rho}{G}{V}$ una rappresentazione, $W\subseteq V$ una sottorappresentazione. Supponiamo che $\ch\mathbb{K}\nmid|G|$. Allora esiste una sottorappresentazione $W'\subseteq V$ tale che $V=W\dirsum W'$.
\end{proposition}
\begin{proof}
Sia $U$ un complementare qualunque di $W$ (cioè tale che $V=W\dirsum U$), e sia $\pi_W:V\to W$ la proiezione. Sia $T:V\to V$ l'applicazione lineare definita da
$$
T=\frac{1}{|G|}\sum_{g\in G}g\pi_Wg^{-1}
$$
. Ricordando che $W$ è $G$-stabile e che $\im\pi_W=\fix\pi_W=W$ si vede facilmente che $\im T\subseteq W$ e che $T|_W=\id$, da cui $\im T=\fix T=W$. Segue che $T^2=T$ e dunque, per decomposizione primaria, $V=V_0\dirsum V_1=\ker T\dirsum\fix T$. Inoltre $T$ è un omomorfismo di rappresentazioni: infatti
\begin{align*}
hT&=\frac{1}{|G|}\sum_{g\in G}hg\pi_Wg^{-1}\\
&=\frac{1}{|G|}\sum_{hg\in G}(hg)\pi_W(hg)^{-1}h\\
&=\frac{1}{|G|}\sum_{g\in G}g\pi_Wg^{-1}h=Th
\end{align*}
. Pertanto, grazie alla \thref{subrepresentation-ker-im}, $\ker T$ è $G$-stabile. Ponendo $W'=\ker T$ si ha, come voluto, $V=W\dirsum W'$  come somma di rappresentazioni.
\end{proof}

\begin{corollary}\thlabel{representation-indecomposable-implies-irreducible}
Sia $G$ un gruppo finito, $\Rep{\rho}{G}{V}$ una rappresentazione indecomponibile. Supponiamo $\ch\mathbb{K}\nmid |G|$. Allora $\rho$ è irriducibile.
\end{corollary}

\begin{corollary}\thlabel{representation-surjective-homomorphism-subrepresentation}
Sia $G$ un gruppo finito, $\Rep{\rho}{G}{V}\comma\Rep{\sigma}{G}{W}$ rappresentazioni, $\varphi\in\Hom_G(V,W)$ un omomorfismo suriettivo. Supponiamo $\ch\mathbb{K}\nmid|G|$. Allora $\sigma$ è una sottorappresentazione di $\rho$.
\end{corollary}
\begin{proof}
Sia $K=\ker\varphi$, e sia $W'\subseteq V$ un complementare $G$-stabile di $K$. Allora $\varphi|_{W'}\in\Hom_G(W',W)$ è un omomorfismo suriettivo e iniettivo, quindi $W'\isor W$.
\end{proof}


\begin{proposition}\thlabel{representation-finite-completely-reducible}
Sia $G$ un gruppo finito, $\Rep{\rho}{G}{V}$ una rappresentazione di grado finito. Supponiamo che $\ch\mathbb{K}\nmid|G|$. Allora $V$ è completamente riducibile.
\end{proposition}
\begin{proof}
Sia $W\subseteq V$ un sottospazio $G$-stabile minimale non nullo. Se $W=V$ non c'è nulla da dimostrare, altrimenti per la \thref{subrepresentation-stable-complement} esiste un complementare $W'$ di $W$ $G$-stabile. Allora $V=W\dirsum W'$ come somma di rappresentazioni con $W$ irriducibile e $\dim W'<\dim V$, dunque si può concludere per induzione.
\end{proof}

\begin{proposition}[Lemma di Schur I]\thlabel{schur-lemma-1}
Siano $\Rep{\rho}{G}{V}\comma\Rep{\sigma}{G}{W}$ rappresentazioni, $\varphi\in\Hom(\rho,\sigma)$.
\begin{enumerate}[(i)]
\item Se $\rho$ è irriducibile, allora $\varphi$ è nullo oppure iniettivo.
\item Se $\sigma$ è irriducibile, allora $\varphi$ è nullo oppure suriettivo.
\item Se $\rho$ e $\sigma$ sono irriducibili, allora $\varphi$ è nullo oppure è un isomorfismo.
\end{enumerate}
\end{proposition}
\begin{proof}\leavevmode
\begin{enumerate}[(i)]
\item $\ker\varphi\subseteq V$ è una sottorappresentazione, quindi è nullo (e allora $\varphi$ è iniettivo) oppure coincide con $V$ (e allora $\varphi$ è nullo).
\item $\im\varphi\subseteq W$ è una sottorappresentazione, quindi è nullo (e allora $\varphi$ è nullo) oppure coincide con $W$ (e allora $\varphi$ è suriettivo).
\item Segue da (i) e (ii).
\end{enumerate}
\end{proof}

\begin{corollary}\thlabel{representation-irreducible-sum-homomorphisms}
Siano $\Rep{\rho_i}{G}{V}$ con $i\in I$ e $\Rep{\sigma_i}{G}{W_j}$ con $j\in J$ rappresentazioni irriducibili tali che $\rho_i\not\isor\sigma_j$ per ogni $i\in I\comma j\in J$ ($I$ e $J$ finiti). Poniamo $\rho=\sum_{i\in I}\rho_i\comma\sigma=\sum_{j\in J}\sigma_j$. Allora $\Hom(\rho,\sigma)=\langle 0\rangle$.
\end{corollary}
\begin{proof}
La tesi segue facilmente dalla \thref{representation-sum-homomorphisms} e dal Lemma di Schur:
\begin{align*}
\Hom(\rho,\sigma)&=\Hom\biggl(\sum_{i\in I}\rho_i,\sum_{j\in J}\sigma_j\biggr)\\
&\iso\Dirsum_{i\in I}\Dirsum_{j\in J}\Hom(\rho_i,\sigma_j)\\
&=\Dirsum_{i\in I}\Dirsum_{j\in J}\langle 0\rangle=\langle 0\rangle
\end{align*}
\end{proof}

\begin{proposition}[Lemma di Schur II]\thlabel{schur-lemma-2}
Sia $\Rep{\rho}{G}{V}$ una rappresentazione irriducibile, $\varphi\in\Hom(\rho,\rho)$. Supponiamo che $\mathbb{K}$ sia algebricamente chiuso. Allora $\varphi$ è un multiplo scalare (eventualmente nullo) dell'identità.
\end{proposition}
\begin{proof}
Sia $\lambda\in\mathbb{K}$ un autovalore per $\varphi$ (esiste poiché $\mathbb{K}$ è algebricamente chiuso). Allora $\varphi-\lambda\id\in\Hom(\rho,\rho)$ non è un isomorfismo (ogni autovettore per $\lambda$ viene mandato in 0), quindi per la \thref{schur-lemma-1} è nullo, ossia $\varphi=\lambda\id$.
\end{proof}

\begin{corollary}\thlabel{representation-center-homothety}
Sia $\Rep{\rho}{G}{V}$ una rappresentazione irriducibile, $z\in Z(G)$. Supponiamo che $\mathbb{K}$ sia algebricamente chiuso. Allora $\rho(z)$ è un multiplo scalare dell'identità.
\end{corollary}
\begin{proof}
Da $z\in Z(G)$ segue $\rho(z)\in\Hom(\rho,\rho)$, quindi per il Lemma di Schur $\rho(z)$ è un multiplo scalare dell'identità.
\end{proof}

\begin{corollary}\thlabel{representation-irreducible-abelian-group}
Sia $G$ un gruppo abeliano, $\Rep{\rho}{G}{V}$ una rappresentazione irriducibile. Supponiamo che $\mathbb{K}$ sia algebricamente chiuso. Allora $\deg\rho=1$.
\end{corollary}
\begin{proof}
Per il \thref{representation-center-homothety} $\rho(g)$ è un multiplo scalare dell'identità per ogni $g\in G$, quindi ogni sottospazio di $V$ è $G$-stabile, pertanto $\dim V=1$.
\end{proof}

\begin{corollary}\thlabel{representation-irreducible-homomorphisms-dim}
Siano $\Rep{\rho}{G}{V}\comma\Rep{\sigma}{G}{W}$ rappresentazioni irriducibili. Supponiamo che $\mathbb{K}$ sia algebricamente chiuso. Allora
$$
\dim\Hom(\rho,\sigma)=
\begin{cases}
1\qquad&\text{se $\rho\isor\sigma$}\\
0\qquad&\text{altrimenti}
\end{cases}
$$
\end{corollary}
\begin{proof}
Se $\rho\not\isor\sigma$ allora per il Lemma di Schur ogni omomorfismo da $\rho$ in $\sigma$ è nullo, quindi $\dim\Hom(\rho,\sigma)=0$. Se invece $\rho\isor\sigma$, sia $\psi$ un isomorfismo da $\rho$ in $\sigma$. Dato $\varphi\in\Hom(\rho,\sigma)$ vale $\psi^{-1}\varphi\in\Hom(\rho,\rho)$, quindi per il Lemma di Schur $\psi^{-1}\varphi$ è un multiplo scalare dell'identità, ovvero $\varphi=\lambda\psi$ per un qualche $\lambda\in\mathbb{K}$. Allora è chiaro che $\Hom(\rho,\sigma)\iso\mathbb{K}$, perciò $\dim\Hom(\rho,\sigma)=1$.
\end{proof}

\begin{corollary}
Siano $\Rep{\rho}{G}{V}\comma\Rep{\sigma}{G}{W}$ rappresentazioni completamente riducibili. Allora $\dim\Hom(\rho,\sigma)=\dim\Hom(\sigma,\rho)$.
\end{corollary}
\begin{proof}
La tesi segue dal \thref{representation-irreducible-homomorphisms-dim} se $\rho$ e $\sigma$ sono irriducibili, e si estende al caso di $\rho\comma\sigma$ completamente riducibili grazie alla \thref{representation-sum-homomorphisms}.
\end{proof}

\begin{proposition}\thlabel{representation-decomposition-formula}
Sia $\Rep{\rho}{G}{V}$ una rappresentazione completamente riducibile. Supponiamo che $\mathbb{K}$ sia algebricamente chiuso. Allora:
\begin{enumerate}[(i)]
\item per ogni $\sigma\in\Irr(G)$ vale $\dim\Hom(\rho,\sigma)<\infty$;
\item l'insieme $\{\sigma\in\Irr(G):\dim\Hom(\rho,\sigma)>0\}$ è finito;
\item vale $\rho=\sum_{\sigma\in\Irr(G)}\dim\Hom(\rho,\sigma)\sigma$;
\end{enumerate}
\end{proposition}
\begin{proof}
Sia $\rho=\sum_{i\in I}n_i\rho_i$ una decomposizione in rappresentazioni irriducibili a due a due non isomorfe, con $n_i\in\mathbb{N^+}$. Sia $\sigma\in\Irr(G)$. Se per ogni $i\in I$ vale $\sigma\not\isor\rho_i$, allora
$$
\dim\Hom(\rho,\sigma)=\sum_{i\in I}n_i\dim\Hom(\rho_i,\sigma)=0
$$
per la \thref{representation-sum-homomorphisms} e il \thref{representation-irreducible-homomorphisms-dim}; questo dimostra la (ii) (dato che $I$ è finito). Se invece esiste un $j\in I$ tale che $\sigma\isor\rho_j$, allora
$$
\dim\Hom(\rho,\sigma)=\sum_{i\in I}n_i\dim\Hom(\rho_i,\sigma)=n_j<\infty
$$
. Questo dimostra la (i). Osserviamo ora che l'uguaglianza in (iii) ha senso (la somma presente al membro di destra è una somma finita di rappresentazioni) e vale
%TODO:l'allineamento fa schifo
\begin{alignat*}{3}
&&\sum_{\sigma\in\Irr(G)}\dim\Hom(\rho,\sigma)\sigma\\
&=&\sum_{\substack{\sigma\in\Irr(G)\\\forall i\in I(\sigma\not\isor\rho_i)}}\dim\Hom(\rho,\sigma)\sigma
&\;&+&\sum_{\substack{\sigma\in\Irr(G)\\\exists j\in I(\sigma\isor\rho_j)}}\dim\Hom(\rho,\sigma)\sigma\\
&=&0\qquad&&+&\sum_{\substack{\sigma\in\Irr(G)\\\exists j\in I(\sigma\isor\rho_j)}}n_j\sigma\\
&=&\sum_{j\in I}n_j\rho_j\qquad&&=&\qquad\rho
\end{alignat*}
\end{proof}

\begin{proposition}\thlabel{representation-decomposition-uniqueness}
Sia $\Rep{\rho}{G}{V}$ una rappresentazione,
$$
\rho=\sum_{i\in I}n_i\sigma_i=\sum_{j\in J}m_j\tau_j
$$
due decomposizioni in rappresentazioni irriducibili a due a due non isomorfe, con $n_i\comma m_j\in\mathbb{N}^+$. Supponiamo che $\mathbb{K}$ sia algebricamente chiuso. Allora esiste una funzione $\theta:J\to I$ biiettiva tale che per ogni $j\in J$ vale $\tau_j\isor\sigma_{\theta(j)}$ e $m_j=n_{\theta(j)}$.
\end{proposition}
\begin{proof}
Sia $j\in J$. Con gli usuali ragionamenti (\thref{representation-sum-homomorphisms} e \thref{representation-irreducible-homomorphisms-dim}) otteniamo che
\begin{align*}
m_j&=\dim\Hom(\rho,\tau_j)\\
&=\dim\Hom\biggl(\sum_{i\in I}n_i\sigma_i,\tau_j\biggr)\\
&=\sum_{i\in I}n_i\dim\Hom(\sigma_i,\tau_j)
\end{align*}
. Poiché $m_j>0$ deve esistere un $i\in I$ tale che $\dim\Hom(\sigma_i,\tau_j)>0$, ovvero (\thref{representation-irreducible-homomorphisms-dim}) tale che $\sigma_i\isor\tau_j$. Tale $i$ è ovviamente unico (le $\sigma_i$ sono a due a due non isomorfe), quindi vale $m_j=n_i$ e possiamo porre $\theta(j)=i$. La funzione $\theta$ così definita è iniettiva (le $\tau_j$ sono a due a due non isomorfe), e per vedere che è suriettiva basta ripetere il ragionamento di sopra con $I$ al posto di $J$. Per come abbiamo definito $\theta$ vale $\tau_j\isor\sigma_{\theta(j)}$, e abbiamo già osservato che $m_j=n_{\theta(j)}$.
\end{proof}

\begin{proposition}\thlabel{representation-decomposition-subrepresentation-uniqueness}
Sia $\Rep{\rho}{G}{V}$ una rappresentazione, $\rho=\sum_{i\in I}n_i\rho_i$ con $n_i\in\mathbb{N}\comma\Rep{\rho_i}{G}{W_i}$ rappresentazioni irriducibili a due a due non isomorfe. Se
$$
\Dirsum_{i\in I}V_i=V=\Dirsum_{i\in I}V_i'
$$
dove $V_i\comma V_i'\subseteq V$ sono sottorappresentazioni tali che $V_i\isor V_i'\isor W_i^{\dirsum n_i}$, allora $V_i=V_i'$ per ogni $i\in I$.
\end{proposition}
\begin{proof}
Fissiamo un $i\in I$. Sia $U_i=\Dirsum\limits_{j\in I\setminus\{i\}}V_j\comma\pi:V\to U_i$ la proiezione indotta dalla decomposizione $V=V_i\dirsum U_i$. Notiamo che $\pi|_{V_i'}\in\Hom(V_i',U_i)$, quindi per il \thref{representation-irreducible-sum-homomorphisms} $\pi|_{V_i'}=0$, da cui $V_i'\subseteq\ker\pi=V_i$. Simmetricamente $V_i\subseteq V_i'$, pertanto $V_i=V_i'$.
\end{proof}

\section{Prodotto di Rappresentazioni}

\begin{definition}
Siano $\Rep{\rho}{G}{V}\comma\Rep{\sigma}{G}{W}$ rappresentazioni. Si definisce \emph{prodotto tensore} (o semplicemente prodotto) di $\rho$ e $\sigma$ la rappresentazione $\rho\tensor \sigma$ (indicata anche con $\rho\sigma$)
\begin{align*}
\rho\tensor\sigma:G&\longrightarrow GL(V\tensor W)\\
g&\longmapsto \rho(g)\tensor\sigma(g)
\end{align*}
\end{definition}

\begin{proposition}\thlabel{representation-tensor-product-properties}
Siano $\Rep{\rho}{G}{V}\comma\Rep{\sigma}{G}{W}\comma\Rep{\tau}{G}{U}$ rappresentazioni.
\begin{enumerate}[(i)]
\item $\deg(\rho\sigma)=\deg\rho\cdot\deg\sigma$
\item $\rho\sigma\isor\sigma\rho$
\item $(\rho\sigma)\tau\isor\rho(\sigma\tau)$
\item $\rho(\sigma+\tau)\isor\rho\sigma+\rho\tau$
\item $n_G\tensor\rho\isor n\rho$ per ogni $n\in\mathbb{N}$ (in particolare $0_G\tensor\rho\isor 0_G$).
\end{enumerate}
\end{proposition}
\begin{proof}
Dimostriamo solo la (ii) e la (v).
\begin{enumerate}[(i)]
\setcounter{enumi}{1}
\item Sia $\psi$ l'applicazione lineare
\begin{alignat*}{3}
\psi:V&\tensor W&&\longrightarrow &W&\tensor V\\
v&\tensor w&&\longmapsto &w&\tensor v
\end{alignat*}
. Sappiamo già (\thref{tensor-product-properties}) che $\psi$ è un isomorfismo di spazi vettoriali; mostriamo che è anche un omomorfismo di rappresentazioni. È sufficiente far vedere che per ogni $g\in G\comma v\in V\comma w\in W$ vale
$$
(\psi\circ\rho\sigma(g))(v\tensor w)=(\sigma\rho(g)\circ\psi)(v\tensor w)
$$
, ma
\begin{align*}
(\psi\circ\rho\sigma(g))(v\tensor w)&=\psi(\rho(g)v\tensor\sigma(g)w)\\
&=\sigma(g)w\tensor\rho(g)v\\
&=\sigma\rho(g)(w\tensor v)\\
&=(\sigma\rho(g)\circ\psi)(v\tensor w)
\end{align*}

\setcounter{enumi}{4}
\item Abbiamo $\Rep{n_G}{G}{\mathbb{K}^n}\comma\Rep{\rho}{G}{V}\comma\Rep{n\rho}{G}{V^{\dirsum n}}$. Sia $\{e_1,\ldots,e_n\}$ la base canonica di $\mathbb{K}^n$. Consideriamo l'applicazione lineare
\begin{alignat*}{3}
\psi:&&\mathbb{K}^n&\tensor V&&\longrightarrow V^{\dirsum n}\\
&&\biggl(\sum_{i=1}^{n}\alpha_ie_i\biggr)&\tensor v&&\longmapsto\alpha_1v\dirsum\ldots\dirsum\alpha_nv
\end{alignat*}
. Si vede facilmente che $\psi$ è un isomorfismo di spazi vettoriali. Per mostrare che è anche un omomorfismo di rappresentazioni, osserviamo che vale
\begin{align*}
(\psi\circ(n_G\tensor\rho)(g))\biggl(\biggl(\sum_{i=1}^{n}\alpha_ie_i\biggr)\tensor v\biggr)
&=\psi\biggl(\biggl(\sum_{i=1}^{n}\alpha_ie_i\biggr)\tensor\rho(g)v\biggr)\\
&=\alpha_1\rho(g)v\dirsum\ldots\dirsum\alpha_n\rho(g)v\\
&=n\rho(g)(\alpha_1v\dirsum\ldots\dirsum\alpha_nv)\\
&=(n\rho(g)\circ\psi)\biggl(\biggl(\sum_{i=1}^{n}\alpha_ie_i\biggr)\tensor v\biggr)
\end{align*}
\end{enumerate}
\end{proof}

\begin{definition}
Sia $n\in\mathbb{N}$ un intero, $\Rep{\rho}{G}{V}$ una rappresentazione.
\begin{itemize}
\item Si definisce \emph{potenza simmetrica $n$-esima} di $\rho$ la rappresentazione
\begin{align*}
\SymP^n\rho:G&\longrightarrow GL(\SymP^nV)\\
g&\longmapsto\SymP^n(\rho(g))
\end{align*}
\item Si definisce \emph{potenza esterna $n$-esima} di $\rho$ la rappresentazione
\begin{align*}
\ExtP^n\rho:G&\longrightarrow GL(\ExtP^nV)\\
g&\longmapsto\ExtP^n(\rho(g))
\end{align*}
\end{itemize}
\end{definition}

\begin{proposition}\thlabel{tensor-squadre-representation-decomposition}
Sia $\Rep{\rho}{G}{V}$ una rappresentazione. Allora $\rho\tensor\rho\isor\SymP^2\rho+\ExtP^2\rho$.
\end{proposition}
\begin{proof}
Sia $\varphi$ l'applicazione lineare
\begin{alignat*}{3}
\varphi:V&\tensor V&&\longrightarrow&\SymP^2V&\dirsum\ExtP^2V\\
v&\tensor w&&\longmapsto& vw&\dirsum v\wedge w
\end{alignat*}
. Sappiamo dalla \thref{tensor-square-decomposition} che $\varphi$ è un isomorfismo di spazi vettoriali; mostriamo che è anche un omomorfismo di rappresentazioni. Per ogni $g\in G\comma v, w\in V$ vale
\begin{align*}
(\varphi\circ\rho\rho(g))(v\tensor w)&=\varphi(\rho(g)v\tensor\rho(g)w)\\
&=(\rho(g)v)(\rho(g)w)\dirsum(\rho(g)v)\wedge(\rho(g)w)\\
&=(\SymP^2\rho(g))(vw)\dirsum(\ExtP^2\rho(g))(v\wedge w)\\
&=(\SymP^2\rho+\ExtP^2\rho)(g)(vw\dirsum v\wedge w)\\
&=((\SymP^2\rho+\ExtP^2\rho)(g)\circ\varphi)(v\tensor w)
\end{align*}
\end{proof}


\section{Rappresentazione Duale}

\begin{definition}
Sia $\Rep{\rho}{G}{V}$ una rappresentazione. Si chiama \emph{rappresentazione duale} la rappresentazione $\Rep{\rho^*}{G}{V^*}$ tale che $\rho^*(g)=\rho(g^{-1})^T$.

\end{definition}
%\begin{remark}
%Vale $\rho^*(g)=(\rho(g)^{-1})^T$. Se $V$ è finitamente generato, chiamiamo $A$ la matrice associata a $\rho(g)$ rispetto a una certa base e $B$ la matrice associata a $\rho^*(g)$ rispetto alla base duale; vale allora $B=(A^{-1})^T$. Se $V$ è anche un $\mathbb{C}$-spazio vettoriale e la base scelta è ortonormale, la relazione diviene $B=\bar{A}$.
%\end{remark}

\begin{proposition}\thlabel{representation-bidual}
Sia $\Rep{\rho}{G}{V}$ una rappresentazione. Allora ${\rho^*}^*\isor\rho$.
\end{proposition}
\begin{proof}
Sia $\Phi:V\to{V^*}^*$ l'isomorfismo canonico (ovvero l'applicazione lineare tale che $\Phi(v)=(f\mapsto f(v))$ per ogni $v\in V$). Verifichiamo che si tratta di un omomorfismo di rappresentazioni. Siano $g\in G,v\in V$; allora
\begin{align*}
{\rho^*}^*(g)\Phi(v)&=\rho^*(g^{-1})^T\Phi(v)=\Phi(v)\circ\rho^*(g^{-1})=\Phi(v)\circ\rho(g)^T\\
&=(f\mapsto\Phi(v)(\rho(g)^Tf))\\
&=(f\mapsto\Phi(v)(f\circ\rho(g)))\\
&=(f\mapsto(f\circ\rho(g))v)\\
&=(f\mapsto f(\rho(g)v))=\Phi(\rho(g)v)
\end{align*}
\end{proof}

\begin{proposition}\thlabel{representation-dual-sum}
Siano $\Rep{\rho}{G}{V}\comma\Rep{\sigma}{G}{W}$ rappresentazioni. Allora $(\rho+\sigma)^*\isor\rho^*+\sigma^*$.
\end{proposition}
\begin{proof}
Consideriamo l'applicazione lineare
\begin{alignat*}{3}
\Theta:(V\dirsum W)^*&\longrightarrow &V^*&\dirsum W^*\\
f&\longmapsto &f\iota_V&\dirsum f\iota_W
\end{alignat*}
dove $\iota_V:V\to V\dirsum W\comma\iota_W:W\to V\dirsum W$ sono le inclusioni. Si vede facilmente che $\Theta$ è invertibile, e l'applicazione inversa è
\begin{alignat*}{2}
\Theta^{-1}:V^*&\dirsum W^*&&\longrightarrow(V\dirsum W)^*\\
f&\dirsum g&&\longmapsto f\pi_V+g\pi_W
\end{alignat*}
dove $\pi_V:V\dirsum W\to V\comma\pi_W:V\dirsum W\to W$ sono le proiezioni. Verifichiamo ora che $\Theta\in\Hom((\rho+\sigma)^*,\rho^*+\sigma^*)$. Siano $g\in G,f\in(V\dirsum W)^*$; allora
\begin{align*}
\Theta((\rho+\sigma)^*(g)f)&=\Theta(f\circ(\rho+\sigma)(g^{-1}))\\
&=f\circ(\rho+\sigma)(g^{-1})\circ\iota_V\dirsum f\circ(\rho+\sigma)(g^{-1})\circ\iota_W\\
&=f\circ\rho(g^{-1})\dirsum f\circ\sigma(g^{-1})\\
&=\rho^*(g)f\circ\iota_V\dirsum\sigma^*(g)f\circ\iota_W=(\rho^*+\sigma^*)(g)\Theta(f)
\end{align*}
\end{proof}

\begin{proposition}\thlabel{representation-dual-irreducible}
Sia $V$ uno spazio vettoriale finitamente generato, $\Rep{\rho}{G}{V}$ una rappresentazione irriducibile. Allora $\Rep{\rho^*}{G}{V^*}$ è irriducibile.
\end{proposition}
\begin{proof}
Sia $W\subseteq V^*$ una sottorappresentazione, e sia $U=\Ann W$. Allora $U$ è una sottorappresentazione di $V$; infatti, dati $g\in G\comma u\in U\comma f\in W$ vale
$$
f(\rho(g)u)=\rho(g^{-1})^T(f)(u)=(\rho^*(g^{-1})f)u=0
$$
poiché $\rho^*(g^{-1})f\in W$; segue che $\rho(g)u\in\Ann(W)=U$, ovvero $U$ è $G$-stabile. Essendo $\rho$ irriducibile, necessariamente $U=\langle 0\rangle$ o $U=V$, da cui $W=V^*$ o $W=\langle 0\rangle$.
\end{proof}

\section{Rappresentazioni per Permutazione e Rappresentazione Regolare}

\begin{definition}
Sia $G$ un gruppo, $X$ un insieme, $\varphi:G\to \Sym(X)$ un'azione. Sia $\mathbb{K}[X]$ lo spazio vettoriale libero su $X$, con $\{e_x\}_{x\in X}$ come base. Si chiama \emph{rappresentazione per permutazione} di $G$ su $X$ la rappresentazione $\Rep{\rho}{G}{\mathbb{K}[X]}$ tale che, per ogni $g\in G$, $\rho(g)$ è l'unica applicazione lineare da $\mathbb{K}[X]$ in sé che fa commutare il diagramma
$$
\begin{tikzcd}
\mathbb{K}[X]\arrow[r,"\rho(g)"]&\mathbb{K}[X]\\
X\arrow[u,"e"]\arrow[r,"\varphi(g)"]&X\arrow[u,"e"]
\end{tikzcd}
$$
, ovvero tale che $\rho(g)e_x=e_{\varphi(g)x}$ per ogni $x\in X$.
\end{definition}

\begin{definition}
Sia $G$ un gruppo. $G$ agisce su $G$ per moltiplicazione a sinistra, ossia $g\cdot x=gx$. Si chiama \emph{rappresentazione regolare} di $G$ la rappresentazione per permutazione $\Rep{\mathcal{R}}{G}{\mathbb{K}[G]}$ associata all'azione di moltiplicazione a sinistra.
\end{definition}

Coerentemente con il già citato abuso di notazione potremo indicare la rappresentazione regolare $\mathcal{R}$ con $\mathbb{K}[G]$. Se $\{e_g\}_{g\in G}$ è la base ``canonica`` di $\mathbb{K}[G]$, la rappresentazione regolare $\mathcal{R}$ agisce in questo modo: per ogni $g,h\in G$ vale $\mathcal{R}(g)e_h=e_{gh}$.

\begin{proposition}\thlabel{representation-regular-homomorphisms-dim}
Sia $G$ un gruppo, $\Rep{\mathcal{R}}{G}{\mathbb{K}[G]}$ la sua rappresentazione regolare, $\Rep{\rho}{G}{V}$ una rappresentazione. Allora 
$$
\dim\Hom(\mathcal{R},\rho)=\deg\rho
$$
\end{proposition}
\begin{proof}
Mostriamo che $\Hom(\mathcal{R},\rho)\iso V$. Consideriamo l'applicazione lineare
$\Phi:\Hom(\mathcal{R},\rho)\to V$ tale che $\Phi(f)=f(e_1)$ per ogni $f\in\Hom(\mathcal{R},\rho)$, dove $1\in G$ è l'identità. $\Phi$ è iniettiva: sia infatti $f\in\Hom(\mathcal{R},\rho)$ tale che $\Phi(f)=f(e_1)=0$. Allora per ogni $g\in G$ vale
$$
f(e_g)=f(e_{g\cdot1})=f(\mathcal{R}(g)e_1)=\rho(g)f(e_1)=\rho(g)0=0
$$
quindi $f=0$. Inoltre $\Phi$ è suriettiva: sia infatti $v\in V$. Sia $f$ l'unica applicazione lineare da $\mathbb{K}[G]$ in $V$ tale che $f(e_g)=\rho(g)v$ per ogni $g\in G$. $f$ è un omomorfismo di rappresentazioni: per ogni $g\in G,u\in V$ vale
\begin{alignat*}{2}
\rho(g)f(u)&=\rho(g)f\biggl(\sum_{h\in G}e_h^*(u)e_h\biggr)
&&=\sum_{h\in G}e_h^*(u)\rho(g)f(e_h)\\
&=\sum_{h\in G}e_h^*(u)\rho(g)\rho(h)v
&&=\sum_{h\in G}e_h^*(u)\rho(gh)v\\
&=\sum_{h\in G}e_h^*(u)f(e_{gh})
&&=\sum_{h\in G}e_h^*(u)f(\mathcal{R}(g)e_h)\\
&=(f\mathcal{R}(g))\biggl(\sum_{h\in G}e_h^*(u)e_h\biggr)&&=f(\mathcal{R}(g)u)
\end{alignat*}
, e ovviamente $\Phi(f)=v$. Dunque $\Phi$ è l'isomorfismo cercato.
\end{proof}

\begin{corollary}\thlabel{representation-regular-decomposition}
Sia $G$ un gruppo, $\Rep{\mathcal{R}}{G}{\mathbb{K}[G]}$ la sua rappresentazione regolare. Supponiamo che $\mathcal{R}$ sia completamente riducibile e che $\mathbb{K}$ sia algebricamente chiuso. Allora $\Irr(G)$ è finito e vale
$$
\mathcal{R}=\sum_{\rho\in\Irr(G)}(\deg\rho)\rho
$$
\end{corollary}
\begin{proof}
La tesi segue immediatamente dalla \thref{representation-decomposition-formula} e dalla \thref{representation-regular-homomorphisms-dim}.
\end{proof}

\begin{remark}
L'ipotesi di completa riducibilità di $\mathcal{R}$ è soddisfatta, ad esempio, se $G$ è un gruppo finito e $\ch\mathbb{K}\nmid|G|$ (\thref{representation-finite-completely-reducible}).
\end{remark}

\begin{corollary}\thlabel{representation-irreducible-degrees}
Sia $G$ un gruppo finito. Supponiamo che $\mathbb{K}$ sia algebricamente chiuso e che $\ch\mathbb{K}\nmid |G|$. Allora
$$
|G|=\sum_{\rho\in\Irr(G)}(\deg\rho)^2
$$
\end{corollary}
\begin{proof}
La tesi segue immediatamente dal \thref{representation-regular-decomposition}.
\end{proof}

\section{Rappresentazioni di Prodotti Diretti}

\begin{definition}
Siano $\Rep{\rho}{G}{V}\comma\Rep{\sigma}{H}{W}$ rappresentazioni. Si definisce \emph{prodotto tensore esterno} di $\rho$ e $\sigma$ la rappresentazione
\begin{alignat*}{2}
\rho\tbox\sigma:G&\times &H&\longrightarrow GL(V\tensor W)\\
(g&,&h)&\longmapsto\rho(g)\tensor\sigma(h)
\end{alignat*}
\end{definition}

\begin{remark}
Se $G=H$, chiamando $\Delta:G\to G\times G$ l'omomorfismo di gruppi tale che $\Delta(g)=(g,g)$ per ogni $g\in G$, vale $(\rho\tbox\sigma)\circ\Delta=\rho\tensor\sigma$.
\end{remark}


\begin{proposition}\thlabel{representation-box-irreducible}
Siano $\Rep{\rho}{G}{V}\comma\Rep{\sigma}{H}{W}$ rappresentazioni irriducibili. Supponiamo che $\mathbb{K}$ sia algebricamente chiuso. Allora $\rho\tbox\sigma$ è irriducibile.
\end{proposition}
\begin{proof}
Premettiamo un Lemma.
\begin{lemma*}
Sia $\langle0\rangle\neq U\subseteq V\tensor W$ un sottospazio $(G\times\langle1\rangle)$-stabile minimale. Allora esiste $\bar{w}\in W$ tale che $U=V\tensor\bar{w}$ (dove con $V\tensor\bar{w}$ si intende $\{v\tensor\bar{w}:v\in V\}$).
\end{lemma*}
\begin{proof}
Sia $\tau$ la rappresentazione
\begin{align*}
\tau:G&\longrightarrow GL(U)\\
g&\longmapsto (\rho\tbox\sigma)(g,1)|_U
\end{align*}
(è ben definita poiché $U$ è $(G\times\langle1\rangle)$-stabile). Per minimalità di $U$, $\tau$ è irriducibile. Sia $\{w_i\}_{i\in I}$ una base di $W$, $\{v_j\}_{j\in J}$ una base di $V$. Poiché $\{v_j\tensor w_i\}_{(i,j)\in I\times J}$ è una base di $V\tensor W$, per ogni $u\in U$ esistono unici $\varphi_i(u)\in V$ con $i\in I$ tali che
$$
u=\sum_{i\in I}\varphi_i(u)\tensor w_i
$$
(osserviamo che solo un numero finito di $\varphi_i(u)$ sono non nulli). Si vede facilmente che $\varphi_i:U\to V$ è un'applicazione lineare per ogni $i\in I$; mostriamo che $\varphi_i\in\Hom(\tau,\rho)$. Per ogni $u\in U$ vale
\begin{align*}
\tau(g)u&=\tau(g)\biggl(\sum_{i\in I}\varphi_i(u)\tensor w_i\biggr)\\
&=\sum_{i\in I}\rho(g)\varphi_i(u)\tensor\sigma(1)w_i\\
&=\sum_{i\in I}\rho(g)\varphi_i(u)\tensor w_i
\end{align*}
, da cui $\varphi_i(\tau(g)u)=\rho(g)\varphi_i(u)$. Osserviamo che i $\varphi_i$ non possono essere tutti nulli, poiché $U$ non è nullo, quindi per il Lemma di Schur almeno uno è un isomorfismo. Per il \thref{representation-irreducible-homomorphisms-dim} $\dim\Hom(\tau,\rho)=1$, quindi esistono $\psi\in\Hom(\tau,\rho)\comma\lambda_i\in\mathbb{K}$ tali che per ogni $i\in I$ vale $\varphi_i=\lambda_i\psi$. $\psi$ non è nullo, dunque è un isomorfismo, ovvero ha nucleo banale. Fissato $0\neq u\in U$, solo un numero finito di $\lambda_i\psi(u)$ sono non nulli, ma $\psi(u)$ è non nullo, pertanto solo un numero finito di $\lambda_i$ sono non nulli. Allora per ogni $u\in U$ vale
$$
u=\sum_{i\in I}\lambda_i\psi(u)\tensor w_i=\psi(u)\tensor\sum_{i\in I}\lambda_i w_i
$$
. Scegliendo $\bar{w}=\sum_{i\in I}\lambda_i w_i$ otteniamo $U=V\tensor\bar{w}$, ovvero la tesi.
\end{proof}
Sia ora $\langle0\rangle\neq U\subseteq V\tensor W$ un sottospazio $(G\times H)$-invariante. Applicando il Lemma a un sottospazio non nullo di $U$ $(G\times\langle1\rangle)$-stabile minimale otteniamo un $0\neq\bar{w}\in W$ tale che $V\tensor\bar{w}\subseteq U$. Per ogni $v\in V$ definiamo $W(v)=\{w\in W:v\tensor w\in U\}$. Si vede facilmente che $W(v)$ è un sottospazio vettoriale di $W$, non nullo poiché contiene $\bar{w}$, e inoltre $H$-stabile: infatti se $h\in H\comma w\in W(v)$ vale
$$
v\tensor\sigma(h)w=(\rho\tbox\sigma)(1,h)(v\tensor w)\in U
$$
essendo $v\tensor w\in U$ e $U$ $(\langle1\rangle\times H)$-stabile. Poiché $\sigma$ è irriducibile, segue che $W(v)=W$ per ogni $v\in V$. Ma allora $\{v\tensor w:v\in V,w\in W\}\subseteq U$, da cui $U=V\tensor W$.
\end{proof}

\begin{corollary}\thlabel{irreducible-representation-direct-product}
Siano $G\comma H$ gruppi finiti. Supponiamo che $\mathbb{K}$ sia algebricamente chiuso e che $\ch\mathbb{K}\nmid|G|\cdot|H|$. Allora
$$
\Irr(G\times H)=\{\rho\tbox\sigma:\rho\in\Irr(G),\sigma\in\Irr(H)\}
$$
\end{corollary}
\begin{proof}
Per la \thref{representation-box-irreducible} sappiamo che $\rho\tbox\sigma$ è irriducibile per ogni $\rho\in\Irr(G)\comma\sigma\in\Irr(H)$; mostriamo che sono tutte diverse. Sia $\iota_G:G\to G\times H$ l'immersione di $G$ in $G\times H$. Osserviamo che per ogni $g\in G$ vale $(\rho\tbox\sigma)\iota_G(g)=\rho(g)\tensor\id$, dunque $(\rho\tbox\sigma)\circ\iota_G\isor(\deg\sigma)\rho$. Segue che $\rho$ (e analogamente $\sigma$) è univocamente determinata da $\rho\tbox\sigma$.\\
Rimane infine da mostrare che tutte le rappresentazioni irriducibili di $G\times H$ sono della forma $\rho\tbox\sigma$ per opportuni $\rho\in\Irr(G)\comma\sigma\in\Irr(H)$. Per il \thref{representation-irreducible-degrees} vale
\begin{align*}
|G\times H|&=|G|\cdot|H|\\
&=\biggl(\sum_{\rho\in\Irr(G)}(\deg\rho)^2\biggr)\biggl(\sum_{\sigma\in\Irr(H)}(\deg\sigma)^2\biggr)\\
&=\sum_{\substack{\rho\in\Irr(G)\\\sigma\in\Irr(H)}}(\deg\rho)^2(\deg\sigma)^2\\
&=\sum_{\substack{\rho\in\Irr(G)\\\sigma\in\Irr(H)}}(\deg\rho\tbox\sigma)^2
\end{align*}
, dunque (sempre per il \thref{representation-irreducible-degrees}) quelle elencate sono tutte le rappresentazioni irriducibili di $G\times H$.
\end{proof}

\section{Rappresentazioni su Omomorfismi}

Supponiamo di avere due rappresentazioni $\Rep{\rho}{G}{V}\comma\Rep{\sigma}{H}{W}$; possiamo costruire in modo naturale una rappresentazione $\tau$ che agisce su $\Hom(V,W)$ nel seguente modo:
\begin{alignat*}{3}
\tau:G&\times&& H&&\longrightarrow GL(\Hom(V,W))\\
(g&,&&h)&&\longmapsto (f\mapsto\sigma(h)\circ f\circ \rho(g)^{-1})
\end{alignat*}
. Se $V$ e $W$ sono finitamente generati, sappiamo (dalla \thref{tensor-homomorphisms-vs-tensor}) che esiste un isomorfismo canonico
\begin{alignat*}{2}
\Psi:V^*&\tensor W&&\longrightarrow\Hom(V,W)\\
\varphi&\tensor w&&\longmapsto(v\mapsto \varphi(v)w)
\end{alignat*}
. Una rappresentazione che agisce naturalmente su $V^*\tensor W$ è $\rho^*\tbox\sigma$. La seguente proposizione mostra che in realtà $\tau$ e $\rho^*\tbox\sigma$ sono la stessa rappresentazione.

\begin{proposition}\thlabel{representation-on-homomorphisms-vs-tensor}
Siano $\rho\comma\sigma\comma\tau\comma\Psi$ come sopra, con l'ipotesi che $V$ e $W$ siano finitamente generati. Allora $\Psi$ è un isomorfismo di rappresentazioni da $\rho^*\tbox\sigma$ in $\tau$.
\end{proposition}
\begin{proof}
È sufficiente mostrare che per ogni $(g,h)\in G\times H\comma f\in V^*\comma w\in W$ vale
$$
\Psi((\rho^*\tbox\sigma)(g,h)(f\tensor w))=\tau(\Psi(f\tensor w))
$$
. Abbiamo
\begin{align*}
\Psi((\rho^*\tbox\sigma)(g,h)(f\tensor w))&=\Psi(\rho^*(g)f\tensor\sigma(h)w)\\
&=\Psi(f\circ\rho(g)^{-1}\tensor\sigma(h)w)\\
&=(v\mapsto f(\rho(g)^{-1}v)\sigma(h)w)\\
&=(v\mapsto \sigma(h)(f(v)w))\circ \rho(g)^{-1}\\
&=\sigma(h)\circ\Psi(f\tensor w)\circ\rho(g)^{-1}\\
&=\tau(\Psi(f\tensor w))
\end{align*}
\end{proof}

Poiché $\Psi$ è un isomorfismo canonico di rappresentazioni, saremo autorizzati a confondere $\rho^*\tbox\sigma$ con $\tau$, considerando la prima come agente indifferentemente su $V^*\tensor W$ o su $\Hom(V,W)$. Nel caso $G=H$ sappiamo che $\rho^*\tensor\sigma=(\rho^*\tbox\sigma)\circ\Delta$, quindi possiamo immaginare che $\rho^*\tensor\sigma$ agisca su $\Hom(V,W)$ come $\tau\circ\Delta$:
\begin{align*}
\rho^*\tensor\sigma:G&\longrightarrow GL(\Hom(V,W))\\
g&\longmapsto(f\mapsto\sigma(g)\circ f\circ\rho(g)^{-1})
\end{align*}

\begin{example}
Sia $G$ un gruppo, e consideriamo l'azione di $G\times G$ su $G$ data da $(g_1,g_2)\cdot g=g_2gg_1^{-1}$. Questa induce una rappresentazione per permutazione $\Rep{\hat{\mathcal{R}}}{G\times G}{\mathbb{K}[G]}$. Supponiamo che $G$ sia finito, che $\mathbb{K}$ sia algebricamente chiuso e che $\ch\mathbb{K}\nmid|G|$. Dimostriamo allora che
$$
\hat{\mathcal{R}}=\sum_{\sigma\in\Irr(G)}\sigma^*\tbox\sigma
$$
. Sappiamo che se $\sigma\in\Irr(G)$, allora $\sigma^*$ è irriducibile (\thref{representation-dual-irreducible}), dunque anche $\sigma^*\tbox\sigma$ è irriducibile (\thref{representation-box-irreducible}); inoltre $\deg\sigma^*\tbox\sigma=(\deg\sigma)^2$, e per il \thref{representation-irreducible-degrees}
$$
\sum_{\sigma\in\Irr}\deg\sigma^*\tbox\sigma=\sum_{\sigma\in\Irr}(\deg\sigma)^2=|G|=\deg\hat{\mathcal{R}}
$$
. Poiché (come già osservato) le rappresentazioni $\sigma^*\tbox\sigma$ sono tutte diverse al variare di $\sigma\in\Irr(G)$, è sufficiente mostrare che $\hat{\mathcal{R}}$ contiene una sottorappresentazione isomorfa a $\sigma^*\tbox\sigma$ per ogni $\sigma\in\Irr(G)$. Fissiamo dunque $\Rep{\sigma}{G}{V_\sigma}$ una rappresentazione irriducibile di $G$. Sia $\{e_g\}_{g\in G}$ la base ``canonica'' di $\mathbb{K}[G]$. Consideriamo l'applicazione lineare
\begin{align*}
T_\sigma:\mathbb{K}[G]&\longrightarrow\End(V_\sigma)\\
e_g&\longmapsto\sigma(g)
\end{align*}
e mostriamo che $T_\sigma\in\Hom(\hat{\mathcal{R}},\sigma^*\tbox\sigma)$:
\begin{align*}
T_\sigma(\hat{\mathcal{R}}(g_1,g_2)e_g)&=T_\sigma(e_{g_2gg_1^{-1}})\\
&=\sigma(g_2gg_1^{-1})\\
&=\sigma(g_2)\sigma(g)\sigma(g_1)^{-1}\\
&=(\sigma^*\tbox\sigma)(g_1,g_2)\sigma(g)\\
&=(\sigma^*\tbox\sigma)(g_1,g_2)T_\sigma(e_g)
\end{align*}
. Osserviamo che $T_\sigma$ non è nulla (ad esempio $T_\sigma(e_1)=\id$) e $\sigma^*\tbox\sigma$ è irriducibile, quindi per il Lemma di Schur $T_\sigma$ è suriettiva. Allora per il \thref{representation-surjective-homomorphism-subrepresentation} $\hat{\mathcal{R}}$ ha una sottorappresentazione isomorfa a $\sigma^*\tbox\sigma$.
\end{example}
