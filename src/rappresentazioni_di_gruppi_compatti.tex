\chapter{Rappresentazioni di Gruppi Compatti}

\section{Gruppi Topologici e Integrale di Haar}
\begin{definition}
Si dice \emph{gruppo topologico} un gruppo $G$ dotato di una topologia tale che:
\begin{itemize}
\item la mappa $(g,h)\mapsto gh$ (da $G\times G$ con la topologia prodotto in $G$) è continua;
\item la mappa $g\mapsto g^{-1}$ (da $G$ in $G$) è continua.
\end{itemize}
\end{definition}
\begin{example}\leavevmode
\begin{itemize}
\item $\mathbb{R}$ con l'addizione e la topologia euclidea è un gruppo topologico.
\item $\mathbb{C}^*$ con la moltiplicazione e la topologia euclidea è un gruppo topologico.
\item Se $G$ è un gruppo topologico e $H\le G$ è un suo sottogruppo, allora $H$ è un gruppo topologico con la topologia di sottospazio. Ad esempio, $S^1\subseteq\mathbb{C}^*$ è un gruppo topologico.
\end{itemize}

\end{example}
\begin{definition}
Siano $G\comma H$ gruppi topologici. Si dice \emph{omomorfismo di gruppi topologici} un omomorfismo di gruppi continuo. Si dice \emph{isomorfismo di gruppi topologici} un isomorfismo di gruppi che è anche un omeomorfismo.
\end{definition}
Per brevità, diremo ``omomorfismo'' per indicare un omomorfismo di gruppi topologici, e ``isomorfismo'' per indicare un isomorfismo di gruppi topologici.
\begin{proposition}
Siano $G\comma H$ gruppi topologici. Allora $G\times H$ con la topologia prodotto è un gruppo topologico.
\end{proposition}
\begin{proof}
Le verifiche sono ovvie.
\end{proof}

\begin{proposition}\thlabel{topological-group-homomorphism-theorem}
Siano $G\comma H\comma K$ gruppi topologici, $f:G\to K$ un omomorfismo, $\pi:G\to K$ un omomorfismo che è anche un'identificazione. Supponiamo che $\ker\pi\subseteq\ker f$. Allora esiste un unico omomorfismo $\bar{f}:H\to K$ che fa commutare il diagramma
$$
\begin{diagram}
G\arrow{r}{f}\arrow[swap]{d}{\pi}&K\\
H\arrow[swap]{ru}{\bar{f}}
\end{diagram}
$$
ovvero tale che $f=\bar{f}\circ\pi$.
\end{proposition}
\begin{proof}
Sia $\bar{f}$ l'unico omomorfismo di gruppi che fa commutare il diagramma; dobbiamo verificare che è continuo. Sia $A\subseteq$ K un aperto; allora $f^{-1}(A)=\pi^{-1}(\bar{f}^{-1}(A))$ è aperto. Poiché $\pi$ è un'identificazione, segue che $\bar{f}^{-1}(A)$ è aperto, ovvero $\bar{f}$ è continua.
\end{proof}

\begin{proposition}\thlabel{topological-group-homomorphism-lifting}
Siano $G\comma H\comma E$ gruppi topologici con $G\comma H$ connessi, $f:H\to G$ un omomorfismo, $p:E\to G$ un omomorfismo che è anche un rivestimento. Supponiamo esista il sollevamento di $f$ a $\bar{f}:H\to E$ tale che $\bar{f}(1)=1$. Allora $\bar{f}$ è un omomorfismo.
$$
\begin{diagram}
\phantom{A}&E\arrow{d}{p}\\
H\arrow{r}{f}\arrow{ru}{\bar{f}}&G
\end{diagram}
$$
\end{proposition}
\begin{proof}
Dati $g,h\in H$ vale
$$
p(\bar{f}(gh))=f(gh)=f(g)f(h)=p(\bar{f}(g))p(\bar{f}(h))=p(\bar{f}(g)\bar{f}(h))
$$
da cui $p(\bar{f}(gh)^{-1}\bar{f}(g)\bar{f}(h))=1$. Consideriamo l'applicazione continua
\begin{alignat*}{2}
\varphi:H&\times &H&\longrightarrow E\\
(g&,&h)&\longmapsto\bar{f}(gh)^{-1}\bar{f}(g)\bar{f}(h)
\end{alignat*}
Notiamo che $\varphi$ è un sollevamento dell'applicazione costante $(g,h)\mapsto 1$, e che $\varphi(1,1)=1$. Per unicità del sollevamento ($H$ è connesso) segue che $\varphi(g,h)=1$ per ogni $g,h\in H$, ossia che $\bar{f}$ è un omomorfismo di gruppi.
\end{proof}

\begin{example}
Classifichiamo tutti gli omomorfismi da $S^1$ in $\mathbb{C}^*$.
\begin{itemize}
\item Per prima cosa cerchiamo gli omomorfismi da $\mathbb{R}$ in $\mathbb{R}$. Osserviamo che tutte le mappe $x\mapsto ax$, con $a\in\mathbb{R}$, sono omomorfismi. Sia ora $f:\mathbb{R}\to\mathbb{R}$ un omomorfismo, e sia $a=f(1)$. Si ricava facilmente $f(x)=ax$ per ogni $x\in\mathbb{Q}$, e si conclude $f(x)=ax$ per ogni $x\in\mathbb{R}$ per continuità di $f$.
\item Cerchiamo ora gli omomorfismi da $S^1$ in $S^1$. Osserviamo che tutte le mappe $z\mapsto z^n$, con $n\in\mathbb{Z}$, sono omomorfismi. Sia ora $f:S^1\to S^1$ un omomorfismo, e sia $\hat{e}:\mathbb{R}\to S^1$ il rivestimento $t\mapsto e^{it}$. Consideriamo il diagramma
$$
\begin{diagram}
\phantom{A}&&\mathbb{R}\arrow{d}{\hat{e}}\\
\mathbb{R}\arrow{r}{\hat{e}}\arrow{rru}{\bar{f}}&S^1\arrow{r}{f}&S^1
\end{diagram}
$$
Poiché $\mathbb{R}$ è semplicemente connesso, esiste $\bar{f}:\mathbb{R}\to\mathbb{R}$ sollevamento di $f\hat{e}$ tale che $\bar{f}(0)=0$; per la \thref{topological-group-homomorphism-lifting} $\bar{f}$ è un omomorfismo. Ma allora $\bar{f}$ è della forma $f(x)=ax$ per un qualche $a\in\mathbb{R}$. Dunque vale $f(\hat{e}(t))=\hat{e}(\bar{f}(t))$, ovvero $f(e^{it})=e^{iat}$ per ogni $t\in\mathbb{R}$. In particolare, ponendo $t=2\pi$ troviamo che $a$ dev'essere intero. Dunque $f$ è della forma $f(e^{it})=e^{int}$ per un qualche $n\in\mathbb{Z}$.
\item Cerchiamo infine gli omomorfismi da $S^1$ in $\mathbb{C}^*$. Osserviamo che gli omomorfismi da $S^1$ in $S^1$ (ossia quelli della forma $z\mapsto z^n$ con $n\in\mathbb{Z}$) sono anche omomorfismi da $S^1$ in $\mathbb{C}^*$; mostriamo che sono tutti. Sia $f:S^1\to\mathbb{C}^*$ un omomorfismo. Notiamo che $f$ manda radici $n$-esime dell'unità in radici $n$-esime dell'unità, che stanno in $S^1$. Poiché le radici dell'unità sono dense in $S^1$, segue che tutta l'immagine di $f$ è contenuta in $S^1$, dunque gli omomorfismi da $S^1$ in $\mathbb{C}^*$ sono esattamente gli omomorfismi da $S^1$ in $S^1$.
\end{itemize}
\end{example}

\begin{definition}
Sia $G$ un gruppo topologico. Si dice \emph{integrale di Haar} un funzionale $\mathbb{R}$-lineare $\int_G:\Cont{G}{\mathbb{R}}\to\mathbb{R}$ che soddisfa le seguenti proprietà:
\begin{enumerate}[(i)]
\item se $f\ge0$ allora $\int_Gf\ge 0$;
\item $\int_G1=1$;
\item sia $f\in\Cont{G}{\mathbb{R}}$, $g\in G$; poniamo $f_g(h)=f(gh)$ per ogni $h\in G$ (osserviamo che $f_g$ è continua); allora $\int_Gf=\int_Gf_g$.
\end{enumerate}
\end{definition}

Se $f\in\Cont{G}{\mathbb{R}}$, potremo scrivere $\int_G f(g)d\mu(g)$ in luogo di $\int_G f$.

Riportiamo senza dimostrazione il seguente teorema.
\begin{proposition}[Haar]\thlabel{haar-integral-existence}
Sia $G$ un gruppo topologico compatto di Hausdorff. Allora esiste un unico integrale di Haar $\int_G$, il quale soddisfa anche, per ogni $f\in\Cont{G}{\mathbb{R}}\comma g\in G$:
$$
\int_Gf(h)d\mu(h)=\int_Gf(hg)d\mu(h)
$$
\end{proposition}
L'integrale di Haar si estende naturalmente a un funzionale $\mathbb{C}$-lineare da $\Cont{G}{\mathbb{C}}$ in $\mathbb{C}$ definendo $\int_G(f_1+if_2)=\int_Gf_1+i\int_Gf_2$.

\begin{example}
Ogni gruppo finito $G$ con la topologia discreta è un gruppo topologico compatto di Hausdorff. L'integrale di Haar di $G$ è dato da
$$
\int_Gf(g)d\mu(g)=\frac{1}{|G|}\sum_{g\in G}f(g).
$$
\end{example}

\section{Rappresentazioni di Gruppi Topologici}
\begin{definition}
Sia $G$ un gruppo topologico, $V$ uno spazio vettoriale complesso finitamente generato. Si dice \emph{rappresentazione continua} un omomorfismo di gruppi topologici $\Rep{\rho}{G}{V}$, dove su $GL(V)$ consideriamo la topologia euclidea.
\end{definition}
Da qui alla fine del capitolo ci limiteremo a considerare gruppi topologici compatti e rappresentazioni continue (e quindi complesse di grado finito). Assumeremo dunque implicitamente le suddette ipotesi.

Rimangono inalterate le definizioni di omomorfismo di rappresentazioni, somma di rappresentazioni\dots.

\begin{proposition}\thlabel{continuous-representations}
Siano $\Rep{\rho}{G}{V}\comma\Rep{\sigma}{G}{W}$ rappresentazioni (continue), $n\in\mathbb{N}$. Allora sono rappresentazioni continue:
\begin{enumerate}[(i)]
\item $\rho+\sigma$;
\item $\rho\tensor\sigma$;
\item $\rho^*$;
\item $\SymP^n\rho$;
\item $\ExtP^n\rho$.
\end{enumerate}
\end{proposition}
\begin{proof}
La continuità delle rappresentazioni sopra elencate può essere verificata componente per componente dal punto di vista matriciale.
\end{proof}

Molti risultati ricavati in generale per le rappresentazioni continuano a valere per rappresentazioni continue. In alcuni casi, però, era stata cruciale l'ipotesi di finitezza del gruppo.

\begin{proposition}\thlabel{compact-representation-stable-hermitian}
Sia $G$ un gruppo compatto, $\Rep{\rho}{G}{V}$ una rappresentazione. Allora esiste una forma hermitiana $\vs{\cdot}{\cdot}:V\times V\to\mathbb{C}$ definita positiva $G$-stabile, ovvero tale che $\vs{v}{w}=\vs{\rho(g)v}{\rho(g)w}$ per ogni $g\in G\comma v,w\in V$.
\end{proposition}
\begin{proof}
Sia $\vs{\cdot}{\cdot}_0:V\times V\to \mathbb{C}$ una qualunque forma hermitiana definita positiva. Fissiamo $v,w\in V$ e osserviamo che l'applicazione $g\mapsto\vs{\rho(g)v}{\rho(g)w}_0$ è continua; possiamo quindi definire
$$
\vs{v}{w}=\int_G\vs{\rho(g)v}{\rho(g)w}_0d\mu(g).
$$
$\vs{\cdot}{\cdot}$ è una forma hermitiana per linearità dell'integrale, ed è $G$-stabile per l'invarianza per traslazione dell'integrale di Haar. Per mostrare che è definita positiva fissiamo un $v\in V$ non nullo. L'applicazione $g\mapsto\vs{\rho(g)v}{\rho(g)v}_0$ è continua, quindi ammette un minimo $l$, che è necessariamente strettamente positivo (dato che $\vs{\cdot}{\cdot}_0$ è definita positiva). Allora $\vs{v}{v}\ge\int_Gl=l>0$.
\end{proof}

\begin{proposition}\thlabel{compact-subrepresentation-stable-complement}
Sia $G$ un gruppo compatto, $\Rep{\rho}{G}{V}$ una rappresentazione, $W\subseteq V$ una sottorappresentazione. Allora esiste una sottorappresentazione $W'\subseteq V$ tale che $V=W\dirsum W'$.
\end{proposition}
\begin{proof}
Sia $\vs{\cdot}{\cdot}$ una forma hermitiana definita positiva $G$-stabile, e sia $W'$ l'ortogonale di $W$. Vale $V=W\dirsum W'$. Per vedere che $W'$ è $G$-stabile consideriamo $w\in W\comma w'\in W'\comma g\in G$. Allora
$$
\vs{w}{\rho(g)w'}=\vs{\rho(g^{-1})w}{w'}=0
$$
poiché $\rho(g^{-1})w\in W$; segue che $\rho(g)w'\in W'$, ossia $W'$ è una sottorappresentazione di $V$.
\end{proof}

\begin{corollary}\thlabel{compact-representation-indecomposable-implies-irreducible}
Sia $G$ un gruppo compatto, $\Rep{\rho}{G}{V}$ una rappresentazione indecomponibile. Allora $\rho$ è irriducibile.
\end{corollary}

\begin{corollary}\thlabel{compact-representation-surjective-homomorphism-subrepresentation}
Sia $G$ un gruppo compatto, $\Rep{\rho}{G}{V}\comma\Rep{\sigma}{G}{W}$ rappresentazioni, $\varphi\in\Hom_G(V,W)$ un omomorfismo suriettivo. Allora $\sigma$ è una sottorappresentazione di $\rho$.
\end{corollary}

\begin{corollary}\thlabel{compact-representation-finite-completely-reducible}
Sia $G$ un gruppo compatto, $\Rep{\rho}{G}{V}$ una rappresentazione. Allora $\rho$ è completamente riducibile.
\end{corollary}

\begin{corollary}\thlabel{compact-representation-eigenvalues}
Sia $G$ un gruppo compatto, $\Rep{\rho}{G}{V}$ una rappresentazione, $g\in G$. Allora gli autovalori di $\rho(g)$ hanno modulo $1$.
\end{corollary}
\begin{proof}
Sia $\vs{\cdot}{\cdot}$ una forma hermitiana definita positiva $G$-stabile, e sia $v$ un autovettore per $\rho(g)$ relativo all'autovalore $\lambda$. Allora
$$
\vs{v}{v}=\vs{\rho(g)v}{\rho(g)v}=\vs{\lambda v}{\lambda v}=|\lambda|^2\vs{v}{v},
$$
da cui $|\lambda|=1$.
\end{proof}

Per quanto riguarda la teoria dei caratteri per gruppi compatti, la \thref{character-properties} rimane vera, in virtù del \thref{compact-representation-eigenvalues} (in realtà la dimostrazione dei caratteri delle potenze simmetriche ed esterne fallisce, ma il risultato continua a valere; la dimostrazione è elementare e poco interessante, dunque non la riportiamo).

Se $G$ è un gruppo compatto, possiamo definire sullo spazio vettoriale $\Cont{G}{\mathbb{C}}$ la forma hermitiana
$$
\vs{f_1}{f_2}_G=\int_Gf_1(g)f_2(g)^*d\mu(g).
$$
È evidente che $\vs{\cdot}{\cdot}_G$ (che, per brevità, indicheremo anche con $\vs{\cdot}{\cdot}$) è semidefinita positiva. Osserviamo che, per ogni rappresentazione $\rho$, $\chi_\rho\in\Cont{G}{\mathbb{C}}$ (infatti $\chi_\rho=\tr\circ\rho$ è composizione di funzioni continue). Vale l'analogo della \thref{character-vs-trivial}.
\begin{proposition}\thlabel{compact-character-vs-trivial}
Sia $G$ un gruppo compatto, $\Rep{\rho}{G}{V}$ una rappresentazione. Allora
$$
\vs{\chi_\rho}{1}=\dim V^G.
$$
\end{proposition}
\begin{proof}
Sia $T:V\to V$ l'applicazione lineare definita da
$$
T=\int_G\rho(g)d\mu(g),
$$
dove l'integrale di endomorfismi si intende svolto componente per componente dal punto di vista matriciale. Dopo aver osservato che, per linearità della traccia e dell'integrale, vale
$$
\tr T=\int_G\tr\rho(g)d\mu(g)=\vs{\chi_\rho}{1},
$$
la dimostrazione è identica a quella della \thref{character-vs-trivial}.
\end{proof}

Dalla \thref{compact-character-vs-trivial} seguono la \thref{character-irreducible-orthogonal} e i Corollari \ref{character-vs-dim-homomorphisms}, \ref{character-representation-decomposition}, \ref{character-representation-isomorphic} e \ref{character-irreducibility-criterion} per gruppi compatti. La \thref{character-irreducible-class-functions-basis} invece non si estende (in modo immediato) al caso di gruppi compatti.

\section{Quaternioni}
\begin{definition}
Si dicono \emph{quaternioni} gli elementi dello spazio vettoriale reale
$$
\mathbb{H}=\mathbb{R}\dirsum\mathbb{R}i\dirsum\mathbb{R}j\dirsum\mathbb{R}k=\{\alpha+\beta i+\gamma j+\delta k:\alpha,\beta,\gamma,\delta\in\mathbb{R}\}
$$
dotato della struttura moltiplicativa definita da
$$
i^2=j^2=k^2=ijk=-1
$$
ed estesa per linearità.
\end{definition}
Possiamo anche scrivere $\mathbb{H}=\mathbb{R}\dirsum\mathbb{R}^3$, dove $\mathbb{R}^3=\mathbb{R}i\dirsum\mathbb{R}j\dirsum\mathbb{R}k$ è l'insieme dei \emph{quaternioni puri}. Ci proponiamo ora di elencare alcune definizioni e proprietà basilari relative ai quaternioni.
\begin{itemize}
\item Dato un quaternione $a=a_0+u$ (si intende che $a_0\in\mathbb{R}$ e $u\in\mathbb{R}^3$), si dice \emph{coniugato} di $a$ il quaternione $a^*=a_0-u$. Osserviamo che $a=a^*$ se e solo se $a$ è reale; inoltre $(ab)^*=b^*a^*$.
\item Per ogni $a\in\mathbb{H}$, $aa^*$ è reale; si può verificare che, in realtà, $aa^*$ è il quadrato della norma euclidea di $a$ visto come elemento di $\mathbb{R}^4$. Definiamo quindi \emph{norma} di $a$ il numero reale $|a|=\sqrt{aa^*}$. La norma è moltiplicativa, ovvero $|ab|=|a||b|$ per ogni $a,b\in\mathbb{H}$.
\item Ogni quaternione $a\in\mathbb{H}$ non nullo è invertibile, con inverso $a^{-1}=\frac{1}{|a|^2}a^*$. Quindi il gruppo $\mathbb{H}^*$ dei quaternioni invertibili coincide con $\mathbb{H}\setminus\{0\}$.
\item $Z(\mathbb{H}^*)=\mathbb{R}^*$. I reali commutano con tutti i quaternioni per definizione. D'altro canto, sia $a=a_0+a_1i+a_2j+a_3k\in Z(\mathbb{H}^*)$; da $ai=ia$ segue facilmente $a_2=a_3=0$, e analogamente da $aj=ja$ segue $a_1=0$, quindi $a=a_0\in\mathbb{R}^*$.
\item Se $u,v\in\mathbb{R}^3$ sono quaternioni puri, vale la seguente formula di moltiplicazione: $uv=-\vs{u}{v}+u\wedge v$ (si intende che $-\vs{u}{v}\in\mathbb{R}$ e $u\wedge v\in\mathbb{R}^3$). In particolare, $uv=vu$ se e solo se $u$ e $v$ sono paralleli, e $uv=-vu$ se e solo se $u$ e $v$ sono ortogonali.
\item $\mathbb{H}$ si immerge in $\mathbb{C}^{2\times 2}$ mediante un omomorfismo di anelli. Possiamo infatti considerare $\mathbb{H}=\mathbb{C}\dirsum\mathbb{C}j=\{a+bj:a,b\in\mathbb{C}\}$ come uno spazio vettoriale complesso di dimensione $2$. Dato $\alpha=a+bj\in\mathbb{H}$ possiamo definire $R_\alpha\in GL(\mathbb{H})$ come $R_\alpha(x)=x\alpha$. Calcolando $R_\alpha(1)=a+bj$ e $R_\alpha(j)=-b^*+a^*j$ troviamo che la matrice associata a $R_\alpha$ rispetto alla base $\{1,j\}$ è $\begin{psmallmatrix}a&-b^*\\b&a^*\end{psmallmatrix}$. Poiché $R_\alpha R_\beta=R_{\beta\alpha}$ conviene passare alla matrice trasposta, ottenendo che l'applicazione
$$
a+bj\longmapsto\begin{pmatrix}a&b\\-b^*&a^*\end{pmatrix}
$$
è un omomorfismo (iniettivo) di anelli.
\end{itemize}

