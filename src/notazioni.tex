\renewcommand{\arraystretch}{1.5}
\newcolumntype{S}{>{\hsize=.2\hsize}X}

\chapter{Notazioni}
\section{Generali}
\begin{tabularx}{\textwidth}{SX}
\hline
$\id$ & dato un insieme $A$ (solitamente chiaro dal contesto), $\id:A\to A$ è la funzione identità (per ogni $a\in A$ vale $\id(a)=a$)\\
$\delta$ & dato un insieme $I$ e un anello $R$ (solitamente chiari dal contesto), $\delta:I\times I\to R$ è la funzione tale che per ogni $i,j\in I$ vale $\delta_{ij}=1$ se $i=j$, $\delta_{ij}=0$ altrimenti \\
$\im f,f(A)$ & $=\{f(a):a\in A\}$ se $f$ è una funzione da un insieme $A$\\
$\fix f$ & $=\{a\in A:f(a)=a\}$ se $f$ è una funzione da $A$ in sé\\
$f|_{A'}$ & se $f:A\to B$ e $A'\subseteq A$, allora $f|_{A'}:A'\to B$ è tale che $f|_{A'}(a)=f(a)$ per ogni $a\in A'$\\
$S(X)$ & il gruppo delle funzioni biiettive da $X$ in sé (l'operazione è la composizione)\\
\hline
\end{tabularx}
\section{Algebra Lineare}
Tutti gli spazi vettoriali si intenderanno definiti su un campo $\mathbb{K}$; eventuali limitazioni su $\mathbb{K}$ (ad esempio $\mathbb{K}=\mathbb{C}$ o $\ch\mathbb{K}=0$) saranno opportunamente segnalate.\\
\begin{tabularx}{\textwidth}{SX}
\hline
$\langle S\rangle$ & se $S\subseteq V$, $\langle S\rangle$ è il sottospazio di $V$ generato dagli elementi di $S$ \\
$V^n$ & $\underbrace{V\dirsum\ldots\dirsum V}_\text{$n$ volte}$\\
$\mathbb{K}^{m\times n}$ & lo spazio vettoriale delle matrici $m\times n$ a coefficienti in $\mathbb{K}$\\
$[A]_{ij}$ & il coefficiente in posizione $(i,j)$ della matrice $A$\\
$\Hom(V,W)$ & lo spazio vettoriale delle applicazioni lineari da $V$ in $W$\\
$\End(V)$ & $=\Hom(V,V)$ \\
$GL(V)$ & il gruppo delle applicazioni lineari invertibili da $V$ in sé (l'operazione è la composizione) \\
0 & dati un insieme $A$ e uno spazio vettoriale $V$, l'applicazione nulla $0:A\to V$ è l'applicazione tale che $0(a)=0_V$ per ogni $a\in A$ \\
$V_\lambda(f)$ & l'autospazio di $f$ relativo all'autovalore $\lambda$ (f può essere omessa se è chiara dal contesto)\\
$V^*$ & lo spazio duale di $V$ (ovvero $V^*=\Hom(V,\mathbb{K})$) \\
$v_i^*$ & se $\{v_i\}_{i\in I}$ è una base di $V$, $v_i^*$ è il funzionale duale di $v_i$, ovvero l'unico elemento di $V^*$ tale che per ogni $j\in I$ vale $v_i^*(v_{j})=\delta_{ij}$\\
$f^T$ & se $f\in\Hom(V,W)$, $f^T\in\Hom(W^*,V^*)$ è l'applicazione lineare tale che $f^T(\varphi)=\varphi\circ f$ per ogni $\varphi\in W^*$; la matrice associata a $f$ rispetto a una certa base è la trasposta della matrice associata a $f^T$ rispetto alla base duale\\
$e_i$ & l'$i$-esimo elemento della base canonica di $\mathbb{K}^n$ (cioè il vettore con coordinate nulle eccetto la $i$-esima, che è 1)\\
$\sum_{i\in I}v_i$ & si intende che solo un numero finito di $v_i$ sono non nulli\\
\hline
\end{tabularx}
