\chapter{Teoria dei Caratteri}
Nella trattazione della teoria dei caratteri ci limiteremo a considerare rappresentazioni complesse di grado finito di gruppi finiti. Per tutto questo capitolo, dunque, assumeremo implicitamente le suddette ipotesi.

\section{Caratteri di Rappresentazioni}

\begin{definition}
Sia $\Rep{\rho}{G}{V}$ una rappresentazione. Si definisce \emph{carattere} di $\rho$ la funzione
\begin{align*}
\chi_\rho:G&\longrightarrow\mathbb{C}\\
g&\longmapsto\tr\rho(g)
\end{align*}
\end{definition}

\begin{proposition}\thlabel{character-properties}
Siano $\Rep{\rho}{G}{V}\comma\Rep{\sigma}{G}{W}$ rappresentazioni, $g,h\in G$. Allora
\begin{enumerate}[(i)]
\item se $\rho\isor\sigma$ allora $\chi_\rho=\chi_\sigma$;
\item $\chi_\rho(1)=\deg(\rho)$;
\item $\chi_\rho(g^{-1})=\chi_\rho(g)^*$;
\item $\chi_\rho(hgh^{-1})=\chi_\rho(g)$;
\item $\chi_{\rho^*}=\chi_\rho^*$;
\item $\chi_{\rho+\sigma}=\chi_\rho+\chi_\sigma$;
\item $\chi_{\rho\sigma}=\chi_\rho\cdot\chi_\sigma$;
\item $\chi_{\SymP^2\rho}(g)=\frac{1}{2}(\chi_\rho(g)^2+\chi_\rho(g^2))$;
\item $\chi_{\ExtP^2\rho}(g)=\frac{1}{2}(\chi_\rho(g)^2-\chi_\rho(g^2))$.
\end{enumerate}
\end{proposition}
\begin{proof}
\leavevmode
\begin{enumerate}[(i)]
\item Sia $\psi:V\to W$ un isomorfismo di rappresentazioni. Allora vale $\psi\rho(g)\psi^{-1}=\sigma(g)$, dunque $\tr\rho(g)=\tr\sigma(g)$ per ogni $g\in G$.
\item $\chi_\rho(1)=\tr(\id)=\dim V=\deg\rho$.
\item Per la \thref{representation-finite-group-diagonalisable} $\rho(g)$ è diagonalizzabile e ha come autovalori solo radici dell'unità, dunque $\chi_\rho(g^{-1})=\tr\rho(g)^{-1}=(\tr\rho(g))^*=\chi_\rho(g)^*$.
\item $\chi_\rho(hgh^{-1})=\tr(\rho(h)\rho(g)\rho(h)^{-1})=\tr\rho(g)=\chi_\rho(g)$.
\item $\chi_{\rho^*}(g)=\tr(\rho(g)^{-1})^T=\tr\rho(g)^{-1}=\tr\rho(g)=\chi_\rho(g)$.
\item $\chi_{\rho+\sigma}(g)=\tr(\rho(g)\dirsum\sigma(g))=\tr\rho(g)+\tr\sigma(g)=\chi_\rho(g)+\chi_\sigma(g)$.
\item Segue dalla \thref{tensor-homomorphism-trace}.
\item Sia $\{v_i\}_{i\in I}$ una base di $V$ di autovettori per $\rho(g)$, con $\rho(g)v_i=\lambda_iv_i$. Allora $\{v_i\}_{i\in I}$ è una base di $V$ di autovettori per $\rho(g)^2$ con $\rho(g)^2v_i=\lambda_i^2v_i$, mentre $\{v_iv_j\}_{i\le j}$ è una base di $\SymP^2V$ di autovettori per $\SymP^2\rho(g)$, con $\SymP^2\rho(g)(v_iv_j)=\lambda_i\lambda_j(v_iv_j)$. Allora
\begin{align*}
\chi_{\SymP^2\rho}(g)&=\tr\SymP^2\rho(g)\\
&=\sum_{i\le j}\lambda_i\lambda_j\\
&=\frac{1}{2}\biggl(\biggl(\sum_{i\in I}\lambda_i\biggr)^2+\sum_{i\in I}\lambda_i^2\biggr)\\
&=\frac{1}{2}((\tr\rho(g))^2+\tr(\rho(g)^2))\\
&=\frac{1}{2}(\chi_\rho(g)^2+\chi_\rho(g^2)).
\end{align*}
\item Identica alla (viii).
\end{enumerate}
\end{proof}

\begin{proposition}\thlabel{character-representation-kernel}
Sia $\Rep{\rho}{G}{V}$ una rappresentazione. Allora
$$
\ker\rho=\{g\in G:\chi_\rho(g)=\deg\rho\}
$$
\end{proposition}
\begin{proof}
\leavevmode
\begin{itemize}
\item[($\subseteq$)] Se $g\in\ker\rho$ allora $\rho(g)=\id$, quindi $\chi_\rho(g)=\tr(\id)=\deg\rho$.
\item[($\supseteq$)] Sia $n=\deg\rho$, e sia $g\in G$ tale che $\chi_\rho(g)\tr\rho(g)=n$. Per la \thref{representation-finite-group-diagonalisable}, $\rho(g)$ è diagonalizzabile e i suoi autovalori hanno modulo $1$; detti $\lambda_1,\ldots,\lambda_n$ gli autovalori di $\rho(g)$, vale $\lambda_1+\ldots+\lambda_n=n$. Ma allora $\lambda_1=\ldots=\lambda_n=1$, quindi $\rho(g)=\id$.
\end{itemize}
\end{proof}


\begin{proposition}\thlabel{character-permutation-representation}
Sia $X$ un insieme, $G$ un gruppo che agisce su $X$, $\Rep{\rho}{G}{\mathbb{C}[X]}$ la corrispondente rappresentazione per permutazione, $g\in G$. Allora $\chi_\rho(g)=|X^g|$.
%$$
%\chi_\rho(g)=|\{x\in X:gx=x\}|.
%$$
\end{proposition}
\begin{proof}
La matrice associata a $\rho(g)$ rispetto alla base ``canonica'' di $\mathbb{C}[X]$ è una matrice di permutazione; i valori sulla diagonale sono 1 se il corrispondente elemento di $X$ viene fissato da $g$, 0 altrimenti, da cui la tesi.
\end{proof}

\begin{corollary}\thlabel{character-regular-representation}
Sia $G$ un gruppo, $\Rep{\mathcal{R}}{G}{\mathbb{C}[G]}$ la sua rappresentazione regolare, $g\in G$. Allora $\chi_\mathcal{R}(g)=|G|\cdot\delta_{1g}$.
\end{corollary}

\section{Ortogonalità}

Dato un gruppo $G$, definiamo una forma hermitiana $\vs{\cdot}{\cdot}_G$ (o più semplicemente $\vs{\cdot}{\cdot}$ se non c'è rischio di ambiguità) sullo spazio vettoriale $\Fun{G}{\mathbb{C}}$:
$$
\vs{f_1}{f_2}_G=\frac{1}{|G|}\sum_{g\in G}f_1(g)f_2(g)^*.
$$
È evidente che $\vs{\cdot}{\cdot}_G$ è definita positiva.

\begin{proposition}\thlabel{character-vs-trivial}
Sia $\Rep{\rho}{G}{V}$ una rappresentazione. Allora
$$
\vs{\chi_\rho}{1}=\dim V^G.
$$
\end{proposition}
\begin{proof}
Sia $T:V\to V$ l'applicazione lineare definita da
$$
T=\frac{1}{|G|}\sum_{g\in G}\rho(g).
$$
Osserviamo che $\im T\subseteq V^G$: infatti, dato $h\in G$, vale
$$
\rho(h)T(v)=\frac{1}{|G|}\sum_{g\in G}\rho(hg)v=\frac{1}{|G|}\sum_{g\in G}\rho(g)v=T(v).
$$
Inoltre $T|_{V^G}=\id$. Segue che $\dim V^G=\tr T$; ma
\begin{align*}
\tr T&=\frac{1}{|G|}\sum_{g\in G}\tr\rho(g)\\
&=\frac{1}{|G|}\sum_{g\in G}\chi_\rho(g)\cdot1\\
&=\vs{\chi_\rho}{1},
\end{align*}
da cui la tesi.
\end{proof}

\begin{proposition}\thlabel{character-irreducible-orthogonal}
Siano $\Rep{\rho}{G}{V}\comma\Rep{\sigma}{G}{W}$ rappresentazioni irriducibili. Allora 
$$
\vs{\chi_\rho}{\chi_\sigma}=
\begin{cases}
1\qquad&\text{se $\rho\isor\sigma$}\\
0\qquad&\text{altrimenti}
\end{cases}.
$$
\end{proposition}
\begin{proof}
Vale
\begin{align*}
\vs{\chi_\rho}{\chi_\sigma}&=\frac{1}{|G|}\sum_{g\in G}\chi_\rho(g)\chi_\sigma(g)^*\\
&=\frac{1}{|G|}\sum_{g\in G}\chi_\rho(g)\chi_{\sigma^*}(g)\\
&=\frac{1}{|G|}\sum_{g\in G}\chi_{\sigma^*\rho}(g)\cdot1\\
&=\vs{\chi_{\sigma^*\rho}}{1}\\
&=\dim(W^*\tensor V)^G
\end{align*}
per la \thref{character-properties} e la \thref{character-vs-trivial}. $G$ agisce su $W^*\tensor V$ mediante $\sigma^*\tensor\rho$, quindi al posto di $W^*\tensor V$ possiamo considerare $\Hom(W,V)$. Applicando la \thref{representation-fix-of-homomorphisms} e il \thref{representation-irreducible-homomorphisms-dim} otteniamo la tesi:
\begin{align*}
\dim(W^*\tensor V)^G&=\dim\Hom(W,V)^G\\
&=\dim\Hom_G(W,V)\\
&=
\begin{cases}
1\qquad&\text{se $\rho\isor\sigma$}\\
0\qquad&\text{altrimenti}
\end{cases}.
\end{align*}
\end{proof}

\begin{corollary}\thlabel{character-vs-dim-homomorphisms}
Siano $\Rep{\rho}{G}{V}\comma\Rep{\sigma}{G}{W}$ rappresentazioni. Allora
$$
\vs{\chi_\rho}{\chi_\sigma}=\dim\Hom(\rho,\sigma)
$$
\end{corollary}
\begin{proof}
La tesi è vera se $\rho\comma\sigma$ sono irriducibili (\thref{character-irreducible-orthogonal}), e si estende a $\rho\comma\sigma$ qualunque per completa riducibilità.
\end{proof}

\begin{corollary}\thlabel{character-representation-decomposition}
Sia $\Rep{\rho}{G}{V}$ una rappresentazione. Allora
$$
\rho=\sum_{\sigma\in\Irr(G)}\vs{\chi_\rho}{\chi_\sigma}\sigma
$$
\end{corollary}
\begin{proof}
Sia $\rho=\sum_{\sigma\in\Irr(G)}n_\sigma\sigma$ ($\rho$ è completamente riducibile). Allora $\chi_\rho=\sum_{\sigma\in\Irr(G)}n_\sigma\chi_\sigma$, dunque per ogni $\tau\in\Irr(G)$ vale
$$
\vs{\chi_\rho}{\chi_\tau}=\sum_{\sigma\in\Irr(G)}n_\sigma\vs{\chi_\sigma}{\chi_\tau}=n_\tau,
$$
da cui la tesi.
\end{proof}

\begin{corollary}\thlabel{character-representation-isomorphic}
Siano $\Rep{\rho}{G}{V}\comma\Rep{\sigma}{G}{W}$ rappresentazioni. Allora $\rho\isor\sigma$ se e solo se $\chi_\rho=\chi_\sigma$.
\end{corollary}
\begin{proof}
Un'implicazione è contenuta nella \thref{character-properties}, l'altra è un'immediata conseguenza del \thref{character-representation-decomposition}.
\end{proof}

\begin{corollary}\thlabel{character-irreducibility-criterion}
Sia $\Rep{\rho}{G}{V}$ una rappresentazione. Allora $\rho$ è irriducibile se e solo se $\vs{\chi_\rho}{\chi_\rho}=1$
\end{corollary}
\begin{proof}
Se $\rho$ è irriducibile allora $\vs{\chi_\rho}{\chi_\rho}=1$ per la \thref{character-irreducible-orthogonal}. Se $\vs{\chi_\rho}{\chi_\rho}=1$, l'irriducibilità segue dal fatto che $\rho$ è completamente riducibile e dal \thref{character-representation-decomposition}.
\end{proof}

\begin{proposition}[Lemma di Burnside]\thlabel{burnside-lemma}
Sia $X$ un insieme finito, $G$ un gruppo (finito) che agisce su $X$. Allora
$$
|X/G|=\frac{1}{|G|}\sum_{g\in G}|X^g|
$$
\end{proposition}
\begin{proof}
Sia $\mathbb{C}[X]$ lo spazio vettoriale libero su $X$ con $\{e_x\}_{x\in X}$ come base. Sia $\Rep{\rho}{G}{\mathbb{C}[X]}$ la rappresentazione per permutazione associata all'azione di $G$ su $X$. Grazie alla \thref{character-permutation-representation} e alla \thref{character-vs-trivial} possiamo scrivere
$$
\frac{1}{|G|}\sum_{g\in G}|X^g|=\frac{1}{|G|}\sum_{g\in G}\chi_\rho(g)=\vs{\chi_\rho}{\chi_1}=\dim\mathbb{C}[X]^G.
$$
Sia $v\in\mathbb{C}[X]$ un generico vettore, e scriviamo
$$
v=\sum_{x\in X}a_xe_x.
$$
Osserviamo che, dato $g\in G$, vale
$$
\rho(g)v=\sum_{x\in X}a_xe_{gx}=\sum_{x\in X}a_{g^{-1}x}e_x.
$$
Ma $v\in\mathbb{C}[X]^G$ se e solo se per ogni $g\in G$ vale $v=\rho(g)v$; notiamo che $v=\rho(g)v$ se e solo se per ogni $x\in X$ vale $a_{gx}=a_x$. Quindi $v\in\mathbb{C}[X]^G$ se e solo se per ogni $x,y$ nella stessa orbita vale $a_x=a_y$. A questo punto si conclude facilmente che la dimensione di $\mathbb{C}[X]^G$ è pari al numero di orbite $|X/G|$.
\end{proof}


\begin{definition}
Sia $G$ un gruppo, $X$ un insieme, $f:G\to X$ una funzione. Si dice che $f$ è una \emph{funzione di classe} se è costante sulle classi di coniugio di $G$, ovvero se per ogni $g,h\in G$ vale $f(hgh^{-1})=f(g)$.
\end{definition}

Dato un gruppo $G$, indichiamo con $\Cl(G)$ il sottospazio vettoriale di $\Fun{G}{\mathbb{C}}$ costituito dalle funzioni di classe. Sappiamo dalla \thref{character-properties} che, per ogni rappresentazione $\rho$, $\chi_\rho\in\Cl(G)$.

\begin{proposition}\thlabel{character-irreducible-class-functions-basis}
Sia $G$ un gruppo. Allora
$$
\mathcal{B}=\{\chi_\rho\}_{\rho\in\Irr(G)}
$$
è una base ortonormale di $\Cl(G)$.
\end{proposition}
\begin{proof}
Ricordiamo che $\Irr(G)$ è finito (\thref{representation-regular-decomposition}).\\
Mostriamo innanzitutto che gli elementi di $\mathcal{B}$ sono indipendenti. Siano $\alpha_\rho\in\mathbb{C}$ coefficienti tali che
$$
0=\sum_{\rho\in\Irr(G)}\alpha_\rho\chi_\rho.
$$
Allora per ogni $\sigma\in\Irr(G)$ vale
$$
\vs{0}{\chi_\sigma}=\biggl\langle\sum_{\rho\in\Irr(G)}\alpha_\rho\chi_\rho,\chi_\rho\biggr\rangle=\alpha_\sigma,
$$
quindi tutti i coefficienti sono nulli. Mostriamo ora che $\mathcal{B}$ è un insieme di generatori; è sufficiente far vedere che $0$ è l'unico vettore ortogonale a tutti gli elementi di $\mathcal{B}$. Sia allora $f\in\Cl(G)$ tale che per ogni $\sigma\in\Irr$ vale $\vs{f}{\chi_\sigma}=0$.

\begin{lemma*}
Sia $\Rep{\rho}{G}{V}$ una rappresentazione. Allora
$$
\sum_{g\in G}f(g)^*\rho(g)=0.
$$
\end{lemma*}
\begin{proof}
Sia $T_f:V\to V$ l'applicazione lineare definita da
$$
T_f=\frac{1}{|G|}\sum_{g\in G}f(g)^*\rho(g).
$$
Verifichiamo che $T_f\in\Hom(\rho,\rho)$: per ogni $h\in G$ vale
\begin{align*}
T_f\rho(h)&=\frac{1}{|G|}\sum_{g\in G}f(g)^*\rho(gh)\\
&=\frac{1}{|G|}\sum_{g\in G}\rho(h)f(g)^*\rho(h^{-1}gh)\\
&=\frac{1}{|G|}\sum_{g\in G}\rho(h)f(h^{-1}gh)^*\rho(h^{-1}gh)\\
&=\frac{1}{|G|}\sum_{g\in G}\rho(h)f(g)^*\rho(g)\\
&=\rho(h)T_f.
\end{align*}
Studiamo il caso in cui $\rho$ è irriducibile. Per il Lemma di Schur $T_f=\lambda\id$ per un qualche $\lambda\in\mathbb{C}$. Applicando la traccia a entrambi i membri otteniamo
\begin{align*}
\lambda\deg\rho&=\frac{1}{|G|}\sum_{g\in G}f(g)^*\chi_\rho(g)\\
&=\frac{1}{|G|}\vs{\chi_{\rho}}{f}\\
&=0.
\end{align*}
Dunque $\lambda=0$, ovvero $T_f=0$. Nel caso generale (cioè se $\rho$ non è necessariamente irriducibile) $\rho$ è completamente riducibile e $T_f$ ristretta alle sottorappresentazioni irriducibili è nulla, quindi $T_f$ è nulla complessivamente.
\end{proof}
Sia $\Rep{\mathcal{R}}{G}{\mathbb{C}[G]}$ la rappresentazione regolare di $G$, e sia $\{e_g\}_{g\in G}$ la base ``canonica'' di $\mathbb{C}[G]$. Applichiamo il Lemma a $\mathcal{R}$: 
$$
0=\sum_{g\in G}f(g)^*\mathcal{R}(g)e_1=\sum_{g\in G}f(g)^*e_g,
$$
quindi $f(g)=0$ per ogni $g\in G$.\\
Dunque $\mathcal{B}$ è una base di $\Cl(G)$, ed è ortonormale per la \thref{character-irreducible-orthogonal}.
\end{proof}

\begin{proposition}\thlabel{character-columns-orthogonal}
Sia $G$ un gruppo, $g,h\in G$. Allora
$$
\sum_{\rho\in\Irr(G)}\chi_\rho(g)\chi_\rho(h)^*=
\begin{cases}
|Z_G(g)|&\text{se $g$ e $h$ sono coniugati}\\
0&\text{altrimenti}
\end{cases}.
$$
\end{proposition}
\begin{proof}
Ricordiamo la rappresentazione $\Rep{\hat{\mathcal{R}}}{G\times G}{\mathbb{C}[G]}$ e la sua decomposizione
$$
\hat{\mathcal{R}}=\sum_{\rho\in\Irr(G)}\rho^*\tbox\rho.
$$
Allora
$$
\sum_{\rho\in\Irr(G)}\chi_\rho(g)\chi_\rho(h)^*=\sum_{\rho\in\Irr(G)}\chi_{\rho^*}(h)\chi_\rho(g)=\sum_{\rho\in\Irr(G)}\chi_{\rho^*\tbox\rho}(h,g)=\chi_{\hat{\mathcal{R}}}(h,g).
$$
Ma $\hat{\mathcal{R}}$ è una rappresentazione per permutazione, quindi
$$
\chi_{\hat{\mathcal{R}}}(h,g)=|\{k\in G:gkh^{-1}=k\}|=|\{k\in G:khk^{-1}=g\}|.
$$
Se $g$ e $h$ non sono coniugati ovviamente $\chi_{\hat{\mathcal{R}}}(h,g)=0$, mentre se lo sono allora $\{k\in G:khk^{-1}=g\}$ è una classe laterale del centralizzatore di $g$ (e di $h$), quindi $\chi_{\hat{\mathcal{R}}}(h,g)=|Z_G(g)|$.
\end{proof}





